\subsection{Aves: complexidade de habitats e diversidade} % Rodolpho
\label{sec:compl-de-habit} 

Até o momento foram realizadas duas das três saídas de campo previstas
para este subprojeto. Na primeira campanha, realizada entre 8 e 30 de
maio, foram feitos o reconhecimento das áreas, teste de dois
potenciais métodos amostrais e a demarcação dos pontos a serem
amostrados nas próximas campanhas. Após uma rápida inspeção do parque
para a escolha de locais potencialmente interessantes para a
realização do projeto, foram testados dois métodos amostrais para o
levantamento da diversidade de aves: método de ponto fixo e método de
transecto. Ambos os métodos se baseiam no registro visual e auditivo
das espécies de aves durante certo intervalo de tempo, porém o método
de ponto fixo consiste no registro das espécies pelo observador
enquanto este se encontra parado em um determinado ponto. No método de
transecto, o observador se locomove por um trajeto (em velocidade
constante) enquanto registra as espécies que visualiza/escuta.

A proposta inicial do projeto era utilizar o método de ponto de
escuta, pois um dos objetivos do projeto, quantificação do uso dos
estratos vegetais pelas aves, depende da visualização direta dos
indivíduos. Nós acreditávamos que esta visualização poderia ser
facilitada se o observador estivesse parado em um mesmo local. No
entanto, o método de transecto também poderia ser mais vantajoso por
permitir a amostragem em uma área amostral maior, visto que observador
não está confinado a uma única localidade. Após 12 dias de amostragem,
foram contabilizados 331 registros de 60 espécies usando o método de
ponto de escuta e 225 registros de 65 espécies utilizando o método de
transecto. No entanto, apenas 18\% de todos os registros feitos
utilizando o método de ponto de escuta foram visuais, sendo que 48\%
de todos os registros foram visuais a partir do método de transecto
(Fig.~\ref{fig:aves1}).  Assim, este resultado nos motivou a escolher o método de
transecto para a continuidade do trabalho.

No total, durante esta primeira campanha, foram realizados 586
registros de 94 espécies, sendo que dentre estes registros alguns
merecem destaque, como por exemplo o da espécie \textit{Urubitinga
coronata} (Águia-cinzenta), que consta na categoria "Em perigo" da
lista nacional de espécies em extinção (MMA, Portaria nº444 de 17 de
dezembro 2014) e também na lista de espécies de extinção de Minas
Gerais (DN COPAM nº147, de 30 de abril de 2010). Outras espécies
observadas que constam na lista de espécies ameaçadas de Minas Gerais
são: \textit{Mycteria americana} (Cabeça-seca, categoria
"Vulnerável"), \textit{Crax fasciolata} (Mutum-de-penacho,categoria
"Em perigo"), \textit{Sporophila angolensis} (Curió, categoria
"Criticamente em Perigo"), \textit{Ara ararauna} (Arara-canindé,
categoria "Vulnerável") e \textit{Ara chloropterus}
(Arara-vermelha-grande, categoria "Criticamente em Perigo").

Além destes registros, foram também demarcados 32 pontos amostrais,
cada um composto de dois transectos de 200 m, distribuídos entre três
fitofisionomias distintas de Cerrado, a saber: campo sujo, campo
cerrado e cerrado \textit{sensu stricto}. Estas fitofisionomias foram
escolhidas por representarem um gradiente de estrutura vegetacional,
que aumenta do campo sujo para o cerrado \textit{sensu stricto}, sendo
o campo cerrado a forma intermediária. Estes pontos amostrais foram
escolhidos de maneira a possibilitar a análise da relação da
diversidade de aves com a estrutura da vegetação no cerrado e foram
amostrados durante a segunda campanha de campo, realizada entre
novembro e dezembro de 2014, e serão amostrados novamente na terceira
campanha, a ser realizada no primeiro semestre de 2015 em datas a
definir.

Na segunda campanha, realizada entre 25 de novembro e 17 dezembro de
2014, foi realizada a primeira etapa das amostragens nos 32 pontos
amostrais selecionados previamente. Durante esta campanha foram
contabilizados 1925 registros de aves de 141 espécies, sendo que
dentre estes foi registrada a espécie \textit{Tigrisoma fasciatum}
(Socó-boi-escuro),considerada como "Vulnerável" em nível nacional
(MMA, Portaria nº444 de 17 de dezembro 2014) e "Criticamente em
Perigo" em Minas Gerais. Outras espécies que constam na lista de
espécies ameaçadas de Minas Gerais e foram registradas neste período
foram: \textit{Culicivora caudacuta}(Papa-moscas-do-campo, categoria
"Vulnerável"), \textit{Euscarthmus rufomarginatus} (Maria-corruíra,
categoria "Criticamente em Perigo"), \textit{Suiriri islerorum}
(Suiriri-da-chapada, categoria "Vulnerável"), além de
\textit{M. americana}, \textit{S. angolensis} e \textit{A. ararauna},
já registradas durante a primeira campanha.

As amostragens foram realizadas nos períodos da manhã (entre 06:00 e
11:00) e à tarde (entre 15:30 e 20:00), evitando os períodos com
incidência de chuva. Metade dos pontos amostrais foram visitados em
pelo menos duas ocasiões pela manhã, cada uma por um observador
diferente, o que irá possibilitar a análise da influência do
observador sobre o registro das espécies e indivíduos e também o
padrão de ocorrência das espécies por fitofisionomia. A outra metade
dos pontos foram visitados entre quatro e seis ocasiões pela manhã e
entre duas e quatro ocasiões à tarde, o que nos permitirá avaliar a
influência do período do dia na detecção das aves, assim como também
calcular a detecção para cada espécie e fisionomia. Em cada registro
foi considerada a presença da ave dentro ou fora da região amostral
(dentro ou fora de raio de 100m em torno do ponto) e o tipo de
registro (visual ou auditivo). Nos registros visuais, foram também
registrados a distância perpendicular das aves em relação ao
transecto, a qual será utilizada para avaliar o efeito da distância de
observação na detecção das aves, e também o estrato vegetacional que o
indivíduo foi avistado, a qual será usada em uma etapa posterior para
avaliação do efeito da complexidade de habitats sobre a diversidade de
espécies. Os estratos vegetacionais de ocorrência dos indivíduos e
espécies foram categorizados da seguinte forma: 1- vegetação herbácea
(0-100cm altura, sem a presença de copa definida e sem a presença de
caule lignificado), 2- vegetação arbustiva (0-100cm altura, sem a
presença de copa definida e com presença de caule lignificado), 3-
vegetação arbórea de pequeno porte (100-300cm de altura, com a
presença de copa definida e caule lignificado) e 4- vegetação arbórea
de porte adulto (a partir de 300cm de altura, com a presença de copa
definida e caule lignificado).

Com relação às três fitofisionomias amostradas, o maior número de
registros foi obtido na fitofisionomia campo cerrado (653), seguido
por campo sujo (640) e cerrado \textit{sensu stricto}
(625). Entretanto, quando consideramos o número de registros
realizados dentro ou fora do raio amostral de 100 m em torno dos
pontos amostrais, os resultados são qualitativamente diferentes.  Os
registros dentro do raio amostral de 100m foram maiores na
fitofisionomia campo sujo (334), depois em campo cerrado (269) e por
último em cerrado \textit{sensu stricto} (214) (Fig.~\ref{fig:aves2}). Já para os
registros realizados fora do raio de 100 m torno dos pontos amostrais,
a fitofisionomia que apresentou maior número de registros foi o
cerrado \textit{sensu stricto} (414), depois o campo cerrado (384) e
por último o campo sujo (306) (Fig.~\ref{fig:aves2}). O maior número de espécies
dentro das áreas amostrais foi registrado na fitofisionomia campo
cerrado (67), seguido por cerrado \textit{sensu stricto} (64) e campo
sujo (60) (Fig.~\ref{fig:aves3}). Entretanto, quando consideramos o número de
espécies registrado fora da áreas amostrais, a fitofisionomia campo
sujo apresentou um maior número de espécies (63), seguido por campo
cerrado (59) e cerrado \textit{sensu stricto} (26) (Fig.~\ref{fig:aves3}).

Estes resultados preliminares indicam que o número de registros feitos
em cada fitofisionomia, assim como o número de espécies presentes em
cada uma, são diferentes. Além disso, estes resultados indicam que
existe um efeito da distância que os registros foram realizados, já
que os registros e espécies encontrados dentro e fora do raio de 100m
em torno dos pontos apresentam diferentes padrões. Este efeito da
distância pode ocorrer tanto por uma maior heterogeneidade de hábitats
em torno de algumas fitofisionomias em relação à outras quanto por um
efeito da detectabilidade associado às características dos hábitats,
já que o efeito positivo da vegetação sobre o número de espécies
poderia ser reduzido ou até mascarado por um efeito negativo da
vegetação sobre a detecção das espécies.

O uso dos estratos pelas aves também foi analisado de maneira
preliminar e dentro do raio de 100m em torno dos pontos foram
realizados 15 registros de aves no estrato 1, 72 no estrato 2, 116 no
estrato 3 e 313 no estrato 4 (Fig.~\ref{fig:aves4}). Com relação ao número de
espécies dentro deste mesmo raio amostral, foram observadas 12
espécies no estrato 1, 28 espécies no estrato 2, 34 no estrato 3 e 63
no estrato 4 (Fig.~\ref{fig:aves4}). Estes resultados mostram que o número de
espécies registradas nos estratos mais baixos é menor que o número de
espécies em registros mais altos, más isto não nos possibilita
concluir que a utilização destes estratos é maior, visto que estes
estratos são também mais visíveis e podem possibilitar uma detecção
maior. Esta proposição parece ter suporte pelos registros fora do raio
amostral de 100m em torno dos pontos amostrais, onde somente sete
registros foram realizados, sendo um registro de uma espécie no
estrato 3 e seis registros de seis espécies diferentes no estrato
4. Assim, quanto maior a distância dos registros, mais dificultada
parece ser a detecção de registros em estratos mais baixos. Ao final
da última campanha de campo, iremos analisar os padrões de diversidade
e ocupação apresentados considerando os efeitos da
detectabilidade. Com isto, esperamos gerar subsídios para alcançar uma
maior compreensão sobre a influência da estrutura e complexidade dos
habitats sobre a diversidade de aves, assim como também um melhor
entendimento dos fatores que podem influenciar a detecção destes
padrões.