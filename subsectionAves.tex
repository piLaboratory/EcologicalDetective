\subsection{Aves: complexidade de habitats e diversidade} % Rodolpho
\label{sec:compl-de-habit} 


Até o momento foram realizadas duas das três saídas de campo previstas para este subprojeto. Na primeira campanha, realizada entre 8 e 30 de maio, foram feitos o reconhecimento das áreas, teste de dois potenciais métodos amostrais e a demarcação dos pontos a serem amostrados nas próximas campanhas. Após uma rápida inspeção do parque para a escolha de locais potencialmente interessantes para a realização do projeto, foram testados dois métodos amostrais para o levantamento da diversidade de aves: método de ponto fixo e método de transecto. Ambos os métodos se baseiam no registro visual e auditivo das espécies de aves durante certo intervalo de tempo, porém o método de ponto fixo consiste no registro das espécies pelo observador enquanto este se encontra parado em um determinado ponto. No método de transecto, o observador se locomove por um trajeto (em velocidade constante) enquanto registra as espécies que visualiza/escuta.
A proposta inicial do projeto era utilizar o método de ponto de escuta, pois um dos objetivos do projeto, quantificação do uso dos estratos vegetais pelas aves, depende da visualização direta dos indivíduos. Nós acreditávamos que esta visualização poderia ser facilitada se o observador estivesse parado em um mesmo local. No entanto, o método de transecto também poderia ser mais vantajoso por permitir a amostragem em uma área amostral maior, visto que observador não está confinado a uma única localidade. Após 12 dias de amostragem, foram contabilizados 331 registros de 60 espécies usando o método de ponto de escuta e 225 registros de 65 espécies utilizando o método de transecto. No entanto, apenas 18\% de todos os registros feitos utilizando o método de ponto de escuta foram visuais, sendo 48\% de todos os registros a partir do método de transecto foram visuais (Figura 1).https://www.authorea.com/users/8982/articles/15869/master/file/figures/dados_piloto1/dados_piloto1.jpg
Assim, este resultado nos motivou a escolher pelo método de transecto para a continuidade do trabalho. 
No total, durante esta primeira campanha foram, realizados 586 registros de 94 espécies, sendo que dentre estes registros alguns merecem destaque, como por exemplo o da espécie \textit{Urubitinga coronata} (Águia-cinzenta), que consta na categoria "Em perigo" da lista nacional de espécies em extinção (MMA, Portaria nº444 de 17 de dezembro 2014) e também na lista de espécies de extinção de Minas Gerais (DN COPAM nº147, de 30 de abril de 2010). Outras espécies observadas que constam na lista de espécies ameaçadas de Minas Gerais são: \textit{Mycteria americana} (Cabeça-seca, categoria "Vulnerável"), \textit{Crax fasciolata} (Mutum-de-penacho,categoria "Em perigo"), \textit{Sporophila angolensis} (Curió, categoria "Criticamente em Perigo"), \textit{Ara ararauna}  (Arara-canindé, categoria "Vulnerável"), \textit{Ara chloropterus} (Arara-vermelha-grande, categoria "Criticamente em Perigo").
Além destes registros, foram demarcados 32 pontos amostrais, cada um composto de dois transectos de 200 m, distribuídos entre três fitofisionomias distintas de Cerrado, a saber: campo sujo, campo cerrado e cerrado \textit{sensu stricto}. Estas fitofisionomias foram escolhidas por representarem um gradiente de estrutura vegetacional, que aumenta do campo sujo para o cerrado \textit{sensu stricto}, sendo o campo cerrado a forma intermediária. Estes pontos amostrais foram amostrados durante a segunda campanha de campo, realizada entre novembro e dezembro de 2014, e serão novamente amostrados na terceira campanha, a ser realizada no primeiro semestre de 2015 em datas a definir.




