\subsection{Aves: complexidade de habitats e diversidade} % Rodolpho
\label{sec:compl-de-habit} 


Até o momento foram realizadas duas das três saídas de campo previstas para este subprojeto. Na primeira campanha, realizada entre 8 e 30 de maio, foram feitos o reconhecimento das áreas, demarcação dos pontos a serem amostrados e o teste dos métodos amostrais. No total, foram demarcados 32 pontos amostrais, distribuídos entre três fitofisionomias distintas de Cerrado, a saber: campo sujo, campo cerrado e cerrado \textit{sensu stricto}. Estas fitofisionomias foram escolhidas por representarem um gradiente de estrutura vegetacional, que aumenta do campo sujo para o cerrado \textit{sensu stricto}, sendo o campo cerrado a forma intermediária. Cada um dos pontos amostrais consiste de dois transectos de 200 m. Também nesta primeira campanha foram testados dois métodos amostrais para o levantamento da diversidade de aves: método de ponto fixo e método de transecto. Ambos os métodos se baseiam no registro visual e auditivo das espécies de aves durante certo intervalo de tempo, porém o método de ponto fixo consiste no registro das espécies pelo observador enquanto este se encontra parado em um determinado ponto. No método de transecto, o observador se locomove por um trajeto (em velocidade constante) enquanto registra as espécies que visualiza/escuta.
Durante esta primeira campanha foram realizados 586 registros de 94 espécies, sendo que dentre estes alguns merecem destaque. A espécie \textit{Urubitinga coronata} (Águia-cinzenta) consta na categoria "Em perigo" da lista nacional de espécies em extinção (MMA, Portaria nº444 de 17 de dezembro 2014) e também na lista de espécies de extinção de Minas Gerais (DN COPAM nº147, de 30 de abril de 2010). Outras espécies que constam na lista de espécies ameaçadas de Minas Gerais são: \textit{Mycteria americana} (Cabeça-seca, categoria "Vulnerável"), \textit{Crax fasciolata} (Mutum-de-penacho,categoria "Em perigo"), \textit{Sporophila angolensis} (Curió, categoria "Criticamente em Perigo"), \textit{Ara ararauna}  (Arara-canindé, categoria "Vulnerável"), \textit{Ara chloropterus} (Arara-vermelha-grande, categoria "Criticamente em Perigo").
Considerando os dois diferntes métodos amostrais empregados nesta primeira campanha, 331 registros de 60 espécies foram realizados usando o método de ponto de escuta e 225 registros de 65 espécies foram feitos utilizando o método de transecto.
A proposta inicial do projeto era utilizar o método de ponto de escuta, pois um dos objetivos do projeto (quantificação do uso dos estratos vegetais pelas aves) depende da visualização dos indivíduos, e nós acreditávamos que esta visualização poderia ser facilitada se o observador estivesse parado em um mesmo local. No entanto, o método de ponto fixo apresentou um baixo percentual do total de registros na forma de contatos visuais (apenas 18\% de todos os contatos), enquanto que o método de transecto se mostrou mais efetivo para a observação e registro visual das aves (48\% de todos os registros)(Figura 1). Assim, este resultado nos motivou a escolher pelo método de transecto para a continuidade do trabalho.



