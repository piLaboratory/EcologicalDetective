\subsection{Aves: complexidade de habitats e diversidade} % Rodolpho
\label{sec:compl-de-habit} 

Até o momento foram realizadas duas das três saídas de campo previstas
para este subprojeto. A primeira viagem estava destinada a estudo piloto
e ajuste do delineamento amostral. Foi realizada entre 8 e 30 de
maio e inclui o reconhecimento das áreas, teste de dois
potenciais métodos amostrais (ponto fixo e transecto) e a definição dos pontos a serem
amostrados nas próximas campanhas. 
A proposta inicial do projeto era utilizar o método de ponto de
fixo escuta, pois um dos objetivos do projeto, quantificação do uso dos
estratos vegetais pelas aves, depende da visualização direta dos
indivíduos. Nós acreditávamos que esta visualização poderia ser
facilitada se o observador estivesse parado em um mesmo local. No
entanto, o método de transecto também poderia ser mais vantajoso por
permitir a amostragem de uma área maior. 
Também definimos os estratos vegetacionais para registro: 
\begin{description}
\item[herbáceo:] formado por capins e ervas de até 100cm altura, sem plantas com copa definida ou caule lignificado;
\item[arbustivo:] arbustos de até 100cm de altura: sem copa definida mas caule lignificado;
\item[arbórea baixa:] arvoretas e árvores de 100 a 300cm de altura, com copa definida;
\item[arbórea:] àrvores acima de 300cm de altura. 
\end{description}

 Após 12 dias de testes em 
pelo menos 5 áreas de cada fitofisionomia (campestre, cerrado, cerradão),
foram contabilizados 331 registros de 60 espécies usando o método de
ponto de escuta e 225 registros de 65 espécies utilizando o método de
transecto. No entanto, apenas 18\% de todos os registros feitos
utilizando o método de ponto de escuta foram visuais, contra 48\%
com  método de transecto. Esse resultado nos motivou a escolher o método de
transecto para a continuidade do trabalho.
Escolhemos então demarcados 32 sítios amostrais,
cada um composto de dois transectos de 200 m, distribuídos entre três
fitofisionomias distintas de Cerrado, a saber: campo sujo, campo
cerrado e cerrado \textit{sensu stricto}. Estas fitofisionomias foram
escolhidas por representarem um gradiente de estrutura vegetacional,
que aumenta do campo sujo para o cerrado \textit{sensu stricto}, sendo
o campo cerrado a forma intermediária. Os sítios foram escolhidos entre
os de acesso viável no parque, distando no mínimo 1 km, buscando-se
evitar agregações de sítios de um só tipo de fisionomia.

Na segunda campanha, realizada entre 25 de novembro e 17 dezembro de
2014, foi realizada a primeira etapa das amostragens nos 32 sítios. 
As amostragens foram realizadas nos períodos da manhã (entre 06:00 e
11:00) e tarde (entre 15:30 e 20:00), evitando os períodos com
incidência de chuva. Metade dos sítios foram visitados em
pelo menos duas ocasiões pela manhã, cada uma por um observador
diferente, o que irá possibilitar a análise da influência do
observador sobre o registro das espécies e também o
padrão de ocorrência das espécies por fitofisionomia. A outra metade
dos sítios foram visitados entre quatro e seis ocasiões pela manhã e
entre duas e quatro ocasiões à tarde, o que nos permitirá avaliar a
influência do período do dia na detecção das aves, assim como também
calcular a detecção para cada espécie e fisionomia. Em cada registro
foi considerada a presença da ave dentro ou fora de raio de 100m em torno do observador
e o tipo de registro (visual ou auditivo). Nos registros visuais, foi
medida a distância perpendicular das aves em relação ao
transecto, a qual será utilizada para avaliar o efeito da distância de
observação na detecção das aves, e também o estrato vegetacional em que o
indivíduo foi avistado. 

Na segunda campanha
contabilizados 1925 registros de aves de 141 espécies.
O maior número de
registros foi obtido na fitofisionomia campo cerrado (653), seguido
por campo sujo (640) e cerrado \textit{sensu stricto}
(625). Entretanto, os
registros dentro do raio amostral de 100m foram mais numerosos na
fitofisionomia campo sujo (334), depois em campo cerrado (269) e por
último em cerrado \textit{sensu stricto} (214) (Fig.~\ref{fig:aves2}). 
Já para os
registros realizados fora do raio de 100m,
o cerrado \textit{sensu stricto} teve o maior número (414), 
seguido do campo cerrado (384) e
por último o campo sujo (306) (Fig.~\ref{fig:aves2}). 
Registramos mais espécies dentro do raio de 100m no
cerrado (67), seguido por cerrado \textit{sensu stricto} (64) e campo
sujo (60). Para fora do raio de 100m 
a fitofisionomia campo sujo apresentou um maior número de espécies (63), seguido por campo
cerrado (59) e cerrado \textit{sensu stricto} (26).
Estes resultados preliminares indicam que o número de registros de indivíduos e de espécies
diferem entre fitofisionomias. Além disso, 
os registros e espécies encontrados dentro e fora do raio de 100m
em torno dos pontos apresentam diferentes padrões. Este efeito da
distância pode ocorrer tanto por uma maior heterogeneidade de hábitats
dentro e em torno de algumas fitofisionomias em relação à outras quanto por um
efeito da detectabilidade associado às características dos hábitats,
já que o efeito positivo da vegetação sobre o número de espécies
poderia ser reduzido ou até mascarado por um efeito negativo da
vegetação sobre a detecção das espécies.

De fato, o efeito das fitofisionomias sobre a ocupação e também sobre
a detecção das espécies já se revela em uma análise preliminar
de ocorrência de três das cinco espécies mais registradas na
primeira campanha. A probabilidade de ocupação do  Tico-tico 
do campo  (\textit{Ammodramus humeralis}, 75 registros) 
diferiu entre três fitofisionomias amostradas
(Fig.~\ref{fig:aves5}), ao contrário de sua detecção. 
Como seu nome popular indica, esta espécie parece ocupar
preferencialmente áreas de campo sujo e campo cerrado, com menor
ocorrência em áreas com mais vegetação mais complexa. Já para o Chibum
(\textit{Elaenia chiriquensis}, 120 registros) e o Tesourão (\textit{Eupetomena
macroura}, 40), as estimativas indicam igual taxa de ocupação mas diferentes 
detectabilidade entre fitofisionomias 
(Fig.~\ref{fig:aves6}). O Chibum teve maior
probabilidade de detecção nas fisionomias campo cerrado e cerrado
\textit{sensu stricto}, pois nessas fisionomias a espécie costuma pousar nos poucos 
poleiros mais altos em busca de presas que passam em vôo. Por sua vez,
\textit{E. macroura} teve maior probabilidade de detecção em cerrado
\textit{sensu stricto}, talvez simplesmente por
maior permanência e consequentemente detecção.
Dentro do raio de 100m em torno dos pontos o número de registros e 
espécies registradas  aumentou
do estrato herbáceo para o arbóreo (Fig.~\ref{fig:aves4}).  
Ainda não acumulamos 
dados o suficiente para fazer a análise de ocupação
por estrato, mas as diferenças podem pelo menos em parte dever-se 
à maior facilidade de detecção de indivíduos pousados
nos arbustos mais altos e árvores. 

Em resumo, as etapas previstas de delineamento e primeira amostragem
foram cumpridas. A julgar pelos registros obtidos até o momento,
será possível estimar diferenças de ocupação das fitofisionomias
pelas espécies mais abundantes e também pela comunidade \cite{dorazio2005}.
Também é possível estimar as riquezas de espécies
em cada fisonomia \cite{dorazio2006}.
Há evidências de que a detecção também varia entre fisionomias,
e o delineamento foi planejado para estimá-las.
Dado que as fitofisionomias diferem marcadamente quanto à complexidade,
a confirmação de diferenças de ocupação de riqueza é um teste
da hipótese de efeito da complexidade estrutural sobre a diversidade
de aves no cerrado. Uma segunda linha de evidência é a estimativa das
taxas de ocupação por estrato, o que depende de uma maior densidade de 
reistros visuais na próxima campanha.