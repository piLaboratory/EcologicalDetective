\subsection{Aves: complexidade de habitats e diversidade} % Rodolpho
\label{sec:compl-de-habit} 


Até o momento foram realizadas duas das três saídas de campo previstas para este subprojeto. Na primeira campanha, realizada entre 8 e 30 de maio, foram feitos o reconhecimento das áreas, teste de dois potenciais métodos amostrais e a demarcação dos pontos a serem amostrados nas próximas campanhas. Após uma rápida inspeção do parque para a escolha de locais potencialmente interessantes para a realização do projeto, foram testados dois métodos amostrais para o levantamento da diversidade de aves: método de ponto fixo e método de transecto. Ambos os métodos se baseiam no registro visual e auditivo das espécies de aves durante certo intervalo de tempo, porém o método de ponto fixo consiste no registro das espécies pelo observador enquanto este se encontra parado em um determinado ponto. No método de transecto, o observador se locomove por um trajeto (em velocidade constante) enquanto registra as espécies que visualiza/escuta.

A proposta inicial do projeto era utilizar o método de ponto de escuta, pois um dos objetivos do projeto, quantificação do uso dos estratos vegetais pelas aves, depende da visualização direta dos indivíduos. Nós acreditávamos que esta visualização poderia ser facilitada se o observador estivesse parado em um mesmo local. No entanto, o método de transecto também poderia ser mais vantajoso por permitir a amostragem em uma área amostral maior, visto que observador não está confinado a uma única localidade. Após 12 dias de amostragem, foram contabilizados 331 registros de 60 espécies usando o método de ponto de escuta e 225 registros de 65 espécies utilizando o método de transecto. No entanto, apenas 18\% de todos os registros feitos utilizando o método de ponto de escuta foram visuais, sendo 48\% de todos os registros a partir do método de transecto foram visuais (Figura 1).
Assim, este resultado nos motivou a escolher o método de transecto para a continuidade do trabalho.