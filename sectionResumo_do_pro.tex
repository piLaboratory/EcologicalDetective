\section{Resumo do projeto}


\subsection{Contexto e objetivo geral}
\label{sec:contexto} 

O desenvolvimento e aplicação de teoria ecológica depende de estimativas
não-enviesadas e precisas da ocorrência, tamanho e taxas vitais de
populações naturais. Há uma percepção crescente de que
tais estimativas são fortemente afetadas por erros de observação, dos
quais as falsas ausências são o exemplo mais óbvio. Assim, o
pressuposto de detecção perfeita ou invariável de espécies e
indivíduos em sido fortemente questionado, bem como os resultados que
dele derivam. Em resposta a esse problema desenvolveu-se um formalismo
de modelos estatísticos que separa dois níveis de
variação dos dados: a camada de observação e a camada do processo
biológico. Os  modelos hierárquicos com detecção
imperfeita são resultado desse formalismo. Têm implementações
computacionais acessíveis a ecólogos, e podem ser aplicadas em muito
mais casos do que têm sido. Por outro lado, essa nova abordagem
envolve procedimentos estatísticos mais complexos e um maior esforço
de amostragem. Assim, é necessário avaliar os ganhos efetivos dos
modelos com detecção imperfeita em diferentes tipos de pesquisas
ecológicas. O objetivo desta proposta é usar estimativas de
ocorrência, abundância e taxas vitais de espécies obtidos com modelos
de detecção imperfeita em três estudos de casos que juntos abrangem
testes sobre estrutura e dinâmicas de populações e comunidades, em
aspectos teóricos e aplicados. Além da importância teórica e aplicada
dos estudos de caso em si, vamos avaliar a sensibilidade dos
resultados e conclusões à adoção da abordagem de modelos com detecção
imperfeita.


\subsection{Objetivos específicos}
\label{sec:especificos} 

\begin{enumerate}
\item Testar se mudanças temporais na abundância relativa e
  diversidade de borboletas nectarívoras de subosque  
  são condizentes com previsões da Teoria Neutra ou com
  previsões de modelos baseados em nicho. 
\item Investigar se a correlações entre a diversidade de estratos da
  vegetação e diversidade de aves de cerrado é explicada pelos padrões
  de uso e sobreposição de uso desses estratos pelas aves.
\item Investigar os efeitos de flutuações climáticas a nível global na
  dinâmica populacional das baleias-jubarte que usam as águas
  brasileiras como área de reprodução.
\item Avaliar o impacto do uso das estimativas com detecção imperfeita
  sobre as conclusões obtidas no três estudos de caso acima.
\end{enumerate}
