\subsection{Baleias: dinâmica populacional e flutuações climáticas} % Leo
\label{sec:dinam-popul-de} 

Entre julho e novembro de 2014 foram realizadas campanhas marítimas embarcadas para contagem e foto-identificação de baleias-jubarte no Banco dos Abrolhos. Contando com o ano de 2014, foram completados 25 anos de série temporal de dados disponíveis para as análises de taxas demográficas e dinâmica populacional dessa espécie em sua área reprodutiva no Brasil. Foram realizadas dez campanhas de campo, totalizando 28 dias de amostragem.

\textbf{Tabela amostragem}: Esforço amostral realizado durante a amostragem da temporada reprodutiva da baelai-jubarte no Banco dos Abrolhos em 2014.

\begin{tabular}{lcccc}
  \textbf{Campanha}        & \textbf{Datas} & \textbf{Dias /amostrados}  & \textbf{Milhas náuticas/ percorridas} &  \textbf{$\Grupos de/ baleia-jubarte} \\   
    01   & 02 a 04 de agosto   & 3   & 129,0   & --   \\            
    02   & 21 a 23 de agosto   & 3   & 126,6   & -- \\
    03   & 27 a 29 de agosto  &  3   & 115,3   & -- \\
    04   & 11 a 13 de agosto  & 3   & 122,1   &  --  \\
    05   & 24 a 26 de setembro  & 3   & 138,1   &  --  \\    
    06   & 08 a 10 de outubro  & 3   & 109,2   &  --  \\
    07   & 24 a 26 de outubro  & 3   & 93,1   &  --  \\
    08   & 01 a 04 de novembro  & 4   & 209,7   &  --  \\
    09   & 07 e 08 de novembro  & 2   & 66,5   &  --  \\
    10   & 12 de novembro  & 1   & 39,2   &  -- \\
    
  \end{tabular}
  
  To be continued soon...