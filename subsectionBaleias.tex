\subsection{Baleias: dinâmica populacional e flutuações climáticas} % Leo
\label{sec:dinam-popul-de} 

Entre julho e novembro de 2014 foram realizadas campanhas marítimas
embarcadas para contagem e foto-identificação de baleias-jubarte no
Banco dos Abrolhos. Considerando o ano de 2014, foram completados 25
anos de série temporal de dados disponíveis para as análises de taxas
demográficas e dinâmica populacional dessa espécie em sua área
reprodutiva no Brasil. A embarcação utilizada foi o \textit{trawler}
Moriá, com casco de madeira medindo cerca de 15 metros de comprimento
(Fig.~\ref{fig:baleia1}).

Foram realizadas dez campanhas de campo, totalizando 28 dias de
amostragem ao longo de quase 1.150 milhas náuticas percorridas em
esforço (Tab.~\ref{tab:baleia1}). Durante a temporada reprodutiva de 2014,
295 grupos de baleias-jubarte foram aproximados pela embarcação,
resultando em um total de 273 baleias-jubarte identificadas através de
marcas naturais na parte ventral da nadadeira caudal (Fig.~\ref{fig:baleia2}). 
Uma análise preliminar revelou pelo menos nove reavistagens de
baleias identificadas em anos anteriores, mas esse número deve
aumentar substancialmente após a realização da comparação do catálogo,
que está em andamento. Amostras de pele para extração de DNA
micro-satélites, que também permite a identificação individual, foram
coletadas para 91 indivíduos. As amostras de tecido para análises
genéticas foram enviadas para análises pelo Laboratório de Biologia
Genômica e Molecular da Pontifícia Universidade Católica do Rio Grande
do Sul, e irão compor o catálogo de identificação genética de baleias.

\begin{table}
  \begin{tabular}{lcccc}  
    \textbf{Campanha} & \textbf{Datas} & \textbf{Dias amostrados} & \textbf{Milhas náuticas percorridas} & \textbf{Grupos de baleia-jubarte} \\
    01 & 02 a 04 de agosto & 3 & 129,0 & 38 \\
    02 & 21 a 23 de agosto & 3 & 126,6 & 39 \\
    03 & 27 a 29 de agosto & 3 & 115,3 & 29 \\
    04 & 11 a 13 de agosto & 3 & 122,1 & 41 \\
    05 & 24 a 26 de setembro & 3 & 138,1 & 34 \\
    06 & 08 a 10 de outubro & 3 & 109,2 & 31 \\
    07 & 24 a 26 de outubro & 3 & 93,1 & 28 \\
    08 & 01 a 04 de novembro & 4 & 209,7 & 29 \\
    09 & 07 e 08 de novembro & 2 & 66,5 & 18 \\
    10 & 12 de novembro & 1 & 39,2 & 08 \\
    \textbf{TOTAL} & & \textbf{28} & \textbf{1.149,9} & \textbf{295} \\
  \end{tabular}
\caption{Esforço amostral realizado durante a temporada reprodutiva da baleia-jubarte no Banco dos Abrolhos em 2014.}
\label{tab:baleia1}    
\end{table}

Análises de uma série temporal de 11 anos de levantamentos aéreos e 22
anos de dados de identificação individual das baleias-jubarte que
reproduzem no Brasil foram realizadas visando estimar as taxas de
crescimento dessa população. Foram utilizados métodos que consideram a
detecção imperfeita para as análises. Para os levantamentos aéreos
foram construídas funções de detecção anuais com os dados de
amostragem de distâncias perpendiculares dos grupos com relação às
linhas de transecção. Essas funções de detecção descrevem como a
probabilidade de detecção de grupos de baleias decresce conforme a
distância da linha. Calcula-se então a densidade de baleias
considerando a probabilidade de detecção em uma faixa determinada de
distância da linha de transecção, resultando em estimativas mais
robustas de abundância. Os dados de foto-identificação foram
analisados utilizados modelos de marcação-recaptura, que permitem o
cálculo de probabilidades de captura para os indivíduos marcados da
população, resultando em estimativas de sobrevivência e taxa de
crescimento mais robustas. Essas análises estão descritas no
manuscrito que se encontra anexo a este relatório. O manuscrito está
em fase final de revisão pelos autores e será submetido durante a
vigência deste projeto de auxílio.