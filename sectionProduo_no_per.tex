\section{Produção no período} % Todos
As instruções da FAPESP para relatórios científicos solicitam a lista de publicações resultantes do projeto no período.
Assim, informo que, entre as modalidades de publicação indicadas nas instruções, foi produzida 


\subsection{Trabalhos preparados} % todos
Também conforme as instruções da FAPESP, incluo esta seção. 


De acordo com as instruções, anexei uma cópia dos manuscritos a este relatório.

\subsection{Participações em congressos} % todos

No período, um dos pesquisadores do projeto (Wedekin, L.L.) participou da XVI Reunião de Especialistas em Mamíferos Aquáticos da América do Sul, que aconteceu em dezembro de 2014 junto com o Congresso de Zoologia da Colômbia e outros eventos na cidade de Cartagena, Colômbia. Na ocasião o referido pesquisador apresentou três trabalhos como co-autor que usaram ou discutiram métodos considerando a detecção imperfeito, sendo eles:

\begin{enumerate}
\item Active and passive acoustics for abundance estimates of Antillean Manatees, \textit{Trichechus manatus}.
\item Primeira estimativa de abundância da população mais austral de botos-cinza, \textit{Sotalia guianensis}, Baía Norte, Sul do Brasil.
\item Estimativa de densidade e abundância de botos, \textit{Sotalia guianensis} (Cetacea: Delphinidae), Baía de Todos os Santos, Bahia, Nordeste do Brasil.
\end{enumerate}

Estes trabalhos tiveram a colaboração de pesquisadores de diferentes instituições brasileiras, como a Universidade Federal do Rio Grande do Norte (1), Universidade Estadual Paulista (1), Universidade Federal do Ceará (1), Universidade Federal de Santa Catarina (2) e Fundação Mamíferos Aquáticos (3).