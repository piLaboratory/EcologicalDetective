\documentclass[12pt, A4]{article}
\usepackage[english,brazil]{babel}
\usepackage[utf8]{inputenc}
\usepackage[T1]{fontenc}
\usepackage{natbib}
\usepackage{url}
\usepackage{graphicx}
\usepackage{fancyhdr,multicol}
\usepackage[paper=a4paper, textwidth=5.5in, top=1in, bottom=1.25in, headheight=.5in, headsep=0.5in, footskip=0.5in]{geometry}
% \setlength{\hoffset}{-0.5in}
% \setlength{\textwidth}{6in}
% \setlength{\voffset}{-0.1in}
% \setlength{\textheight}{10in}
% \setlength{\parindent}{0pt}
\pdfpagewidth=\paperwidth
\pdfpageheight=\paperheight
\setlength{\parskip}{1ex plus 0.5ex minus 0.2ex} % espaco entre paragrafos e retira tabulacao no inicio
\newcommand{\R}{{\sf R}}
\newcommand{\code}[1]{\texttt{#1}}
\renewcommand{\footrulewidth}{0.5pt}
\begin{document}
%\maketitle
\pagestyle{fancyplain}
\lhead[] {\textsl{\footnotesize } \\ \textsl{\footnotesize Relatório Proc. FAPESP 2013/19250--7}}
\rhead[\fancyplain{}{\textsl{\thepage}}]%
      {\fancyplain{}{\textsl{\thepage}}}
\rfoot[]{\textsl{\footnotesize Laboratório de Ecologia Teórica -- IBUSP} \\
  \footnotesize \url{http://ecologia.ib.usp.br/let}}
\lfoot[]{}
\cfoot[]{}
\thispagestyle{empty}
%%%%%%%%%%%%%%%%%%%%%%%%%%%%%%%%%%%%%%%%%%%%%%%%%%%%%%%%%%%%%%%%%%%%%%%%%%%%%%%%%%%%%%%
%\vspace{-10in}
\begin{center}
  {\LARGE Relatório Científico de Andamento} 
\end{center}
\begin{description}
\item[Título do Projeto:] Ferramentas do Detetive Ecológico: uso e avaliação de modelos com detecção imperfeita
\item[Responsável:] Paulo Inácio de Knegt López de Prado
\item[Instituição-sede:] Instituto de Biociências, Universidade de São Paulo
\item[Equipe Executora]{\bf:}
  \begin{itemize}
  \item Carlos Ernesto Candia-Gallardo, Pós-graduação em Ecologia IB--USP
  \item Leonardo Liberali Wedekin, Pós-doutorado IB--USP
  \item Rodolpho Credo Rodrigues, Pós-graduação em Ecologia IB--USP
  \item Karlla Barbosa dos Santos, Bolsista TT-III
  \item Melina de Souza Leite, Especialista em Laboratório, IB--USP
  \end{itemize}
\item[Processo:] 2013/19250-7 Auxílio Pesquisa - Regular 
\item[Vigência:] 01/02/2014 a 31/01/2016 
\item[Período deste relatório:] 01/02/2014 a 30/01/2015
\end{description}

\section{Resumo do projeto}

\subsection{Contexto e objetivo geral}
\label{sec:contexto}
O desenvolvimento e aplicação de teoria ecológica depende de estimativas
não-enviesadas e precisas da ocorrência, tamanho e taxas vitais de
populações naturais. Há uma percepção crescente de que
tais estimativas são fortemente afetadas por erros de observação, dos
quais as falsas ausências são o exemplo mais óbvio. Assim, o
pressuposto de detecção perfeita ou invariável de espécies e
indivíduos em sido fortemente questionado, bem como os resultados que
dele derivam. Em resposta a esse problema desenvolveu-se um formalismo
de modelos estatísticos que separa dois níveis de
variação dos dados: a camada de observação e a camada do processo
biológico. Os  modelos hierárquicos com detecção
imperfeita são resultado desse formalismo. Têm implementações
computacionais acessíveis a ecólogos, e podem ser aplicadas em muito
mais casos do que têm sido. Por outro lado, essa nova abordagem
envolve procedimentos estatísticos mais complexos e um maior esforço
de amostragem. Assim, é necessário avaliar os ganhos efetivos dos
modelos com detecção imperfeita em diferentes tipos de pesquisas
ecológicas. O objetivo desta proposta é usar estimativas de
ocorrência, abundância e taxas vitais de espécies obtidos com modelos
de detecção imperfeita em três estudos de casos que juntos abrangem
testes sobre estrutura e dinâmicas de populações e comunidades, em
aspectos teóricos e aplicados. Além da importância teórica e aplicada
dos estudos de caso em si, vamos avaliar a sensibilidade dos
resultados e conclusões à adoção da abordagem de modelos com detecção
imperfeita.

\subsection{Objetivos específicos}
\label{sec:especificos}

\begin{enumerate}
\item Testar se mudanças temporais na abundância relativa e
  diversidade de borboletas nectarívoras de subosque  
  são condizentes com previsões da Teoria Neutra ou com
  previsões de modelos baseados em nicho. 
\item Investigar se a correlações entre a diversidade de estratos da
  vegetação e diversidade de aves de cerrado é explicada pelos padrões
  de uso e sobreposição de uso desses estratos pelas aves.
\item Investigar os efeitos de flutuações climáticas a nível global na
  dinâmica populacional das baleias-jubarte que usam as águas
  brasileiras como área de reprodução.
\item Avaliar o impacto do uso das estimativas com detecção imperfeita
  sobre as conclusões obtidas no três estudos de caso acima.
\end{enumerate}

\section{Resultados do período}
% Cada seção descrever brevemente:
% 1- Resumo das atividades de coleta de dados realizadas (descrição
% breve das viagens/coletas e o resultado geral de cada uma, uma
% tabela de resumo dos dados obtidos até o momento). 
% 2 - Mudanças de metodologia e/ou no cronograma, justificadas
% 3 - Mudanças e acréscimos de objetivos, com justificativa
% 4 - Próximos passos, com referência explícita ao previsto no cronograma
% 5 - outras observações que julgar pertinentes

\subsection{Borboletas: dinâmica de comunidades e neutralidade} %Kiwi
\label{sec:dinamica-temporal-borb}
%% Kiwi: pode ater-se à pergunta original deste projeto, mesmo que em
%% seu doutorado estejamos em mudança. Os dados permitem testar a
%% hipótese original e por ora creio que basta mostrar que estamos
%% coletando-os adequadamente. O resultado parcial que pode ser
%% adiantado é a comparação das taxas demográficas entre as diversas espécies.

\subsection{Aves: complexidade de habitats e diversidade} % Rodolpho
\label{sec:compl-de-habit}
%% Descrição das atividades do piloto
%% Descrição do delineamento amostral
%% Resumo dos dados coletados até ao momento: um resumo que indique
%% que a densidade de registros é suficiente para responder às perguntas

\subsection{Baleias: dinâmica populacional e flutuações climáticas} % Leo
\label{sec:dinam-popul-de}
%% Além dos resultados e análise já realizados, indicar rede de
%% colaborações estabelecidas e como se articulam com o projeto.

\subsection{Síntese: sensibilidade à detecção imperfeita} % PI

\section{Conclusões} %PI


\section{Apoio institucional} %PI
Recebi todo o apoio institucional necessário, que foi detalhado no formulário de infraestrutura institucional, 
encaminhado com o termo de outorga. 

\section{Aplicação da RT e benefícios complementares} %PI

% Utilizei a maior parte da reserva técnica e benefícios complementares na adequação do laboratório que recebi
% de meu departamento. Foram necessárias mudanças de instalação, remoção de bancadas e aquisição de mobiliário
% para torná-lo um ambiente adequado ao trabalho teórico e computacional. Antes de realizar estes 
% gastos, consultei o canal \emph{``Converse com a FAPESP''}, e recebi autorização. 
% Usei uma pequena fração da reserva técnica para comprar uma mochila para \emph{notebook},
% e cerca de R\$3.150,00 em mais um \emph{notebook} para a equipe.

\section{Produção no período} % Todos
%indiquem se houve algum artigo, comunicação em congresso, etc
As instruções da FAPESP para relatórios científicos solicitam a lista de publicações resultantes do projeto no período.
Assim, informo que, entre as modalidades de publicação indicadas nas instruções, foi produzida 


\subsection{Trabalhos preparados} % todos
%% Manuscritos em preparação derivados od projeto: indicar aqui e
%% enviar cópia a mim ou incluí-la no GitHub
Também conforme as instruções da FAPESP, incluo esta seção. 

De acordo com as instruções, anexei uma cópia dos manuscritos a este relatório.

\bibliographystyle{ecol_let}

\end{document}
