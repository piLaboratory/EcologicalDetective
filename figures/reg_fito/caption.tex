\textbf{Figura 2}: Número de registros de aves realizados em cada uma das fitofisionomias amostradas, considerando os rgistros feitos dentro do raio amostral de 100m e fora deste raio amostral. Legenda: CS- Campo sujo; CC -  Campo Cerrado; SS- Cerrado \textit{Sensu stricto}. Note que o comprimento do eixo y é diferente nos dois gráficos.

O maior número de espécies dentro das áreas amostrais foi registrado na fitofisionomia campo cerrado (67), seguido por cerrado \textit{sensu stricto} (64) e campo sujo (60) (Figura \ref{fig:Figura 3}). Entretanto, quando consideramos o número de espécies registrado fora da áreas amostrais, a fitofisionomia campo sujo apresentou um maior número de espécies (63), seguido por campo cerrado (59) e cerrado \textit{sensu stricto} (26) (Figura \ref{fig:Figura 3}).
