\textbf{Figura 2.1-5}: Similaridade de Bray-Curtis das amostras mensais de Ithomiini com o centroide da comunidade, o qual foi obtido calculando-se o número médio de indivíduos de cada espécie capturados por mês.

A curva-espécies tempo baseada em indivíduos obtida no Parque Esporte para Todos de abril/13 a out/14 está representada na figura 2.1-6. É possível notar uma tendência à estabilização da curva indicando que as espécies mais comuns da comunidade foram registradas e que novas espécies que eventualmente podem ser capturadas devem ser representadas por indivíduos vagantes.

Figura
Figura 2.1-6. Curva espécies-tempo baseada em indivíduos obtida no Parque Esporte Para Todos entre abril/13 e outubro/14. OS dados foram aleatorizados 100 vezes e as linhas mais estreitas representam o intervalo de confiança de 95%.

Nas próximas etapas do projeto os dados empíricos (DAE, turnover e curvas-espécies tempo) serão comparados com dados simulados sob um cenário de neutralidade. Ainda, serão explorados parâmetros demográficos das diferentes espécies (obtidos pela técnica de captura-marcação-recaptura) a fim de testar a premissa de equivalência demográfica evocada pela Teoria Neutra. Também serão testadas hipóteses a respeito de equivalência demográfica dentro – porém não entre – anéis miméticos.