\textbf{Figura 1}: Proporção de registros auditivos e visuais realizados a partir dos métodos de ponto de escuta e transecto, respectivamente. Note que o comprimento do eixo y é diferente nos dois gráficos.

No total, durante esta primeira campanha, foram realizados 586 registros de 94 espécies, sendo que dentre estes registros alguns merecem destaque, como por exemplo o da espécie \textit{Urubitinga coronata} (Águia-cinzenta), que consta na categoria "Em perigo" da lista nacional de espécies em extinção (MMA, Portaria nº444 de 17 de dezembro 2014) e também na lista de espécies de extinção de Minas Gerais (DN COPAM nº147, de 30 de abril de 2010). Outras espécies observadas que constam na lista de espécies ameaçadas de Minas Gerais são: \textit{Mycteria americana} (Cabeça-seca, categoria "Vulnerável"), \textit{Crax fasciolata} (Mutum-de-penacho,categoria "Em perigo"), \textit{Sporophila angolensis} (Curió, categoria "Criticamente em Perigo"), \textit{Ara ararauna}  (Arara-canindé, categoria "Vulnerável") e \textit{Ara chloropterus} (Arara-vermelha-grande, categoria "Criticamente em Perigo").

Além destes registros, foram também demarcados 32 pontos amostrais, cada um composto de dois transectos de 200 m, distribuídos entre três fitofisionomias distintas de Cerrado, a saber: campo sujo, campo cerrado e cerrado \textit{sensu stricto}. Estas fitofisionomias foram escolhidas por representarem um gradiente de estrutura vegetacional, que aumenta do campo sujo para o cerrado \textit{sensu stricto}, sendo o campo cerrado a forma intermediária. Estes pontos amostrais foram escolhidos de maneira a possibilitar a análise da relação da diversidade de aves com a estrutura da vegetação no cerrado e foram amostrados durante a segunda campanha de campo, realizada entre novembro e dezembro de 2014, e serão amostrados novamente na terceira campanha, a ser realizada no primeiro semestre de 2015 em datas a definir.

As amostragens foram realizadas nos períodos da manhã (entre 06:00 e 11:00) e à tarde (entre 15:30 e 20:00), evitando os períodos com incidência de chuva. Metade dos pontos amostrais foram visitados em pelo menos duas ocasiões pela manhã, cada uma por um observador diferente, o que irá possibilitar a análise da influência do observador sobre o registro das espécies e indivíduos e também o padrão de ocorrência das espécies por fitofisionomia. A outra metade dos pontos foram visitados entre quatro e seis ocasiões pela manhã e entre duas e quatro ocasiões à tarde, o que nos permitirá avaliar a influência do período do dia na detecção das aves, assim como também calcular a detecção para cada espécie e fisionomia. Em cada registro foi considerada a presença da ave dentro ou fora da região amostral (dentro ou fora de raio de 100m em torno do ponto) e o tipo de registro (visual ou auditivo). Nos registros visuais, foram também registrados a distância perpendicular das aves em relação ao transecto, a qual será utilizada para avaliar o efeito da distância de observação na detecção das avess, e também o estrato vegetacional que o indivíduo foi avistado, a qual será usada em uma etapa posterior para avaliação do efeito da complexidade de habitats sobre a diversidade de espécies. Os estratos vegetacionais de ocorrência dos indivíduos e espécies foram categorizados da seguinte forma: 1- vegetação herbácea (0-100cm altura, sem a presença de copa definida e sem a presença de caule lignificado), 2- vegetação arbustiva (0-100cm altura, sem a presença de copa definida e com presença de caule lignificado), 3- vegetação arbórea de pequeno porte (100-300cm de altura, com a presença de copa definida e caule lignificado) e 4- vegetação arbórea de porte adulto (a partir de 300cm de altura, com a presença de copa definida e caule lignificado).

Com relação às três fitofisionomias amostradas, o maior número de registros foi obtido na fitofisionomia campo cerrado (653), seguido por campo sujo (640) e cerrado \textit{sensu stricto} (625). Entretanto, quando consideramos o número de registros realizados dentro ou fora do raio amostral de 100 m em torno dos pontos amostrais, os resultados são qualitativamente diferentes.
Os registros dentro do raio amostral de 100m foram maiores na fitofisionomia campo sujo (334), depois em campo cerrado (269) e por último em cerrado \textit{sensu stricto} (Figura \ref{fig:Figura 2}) (214). Já para os registros realizados fora do raio de 100 m torno dos pontos amostrais, a fitofisionomia que apresentou maior número de registros foi o cerrado \textit{sensu stricto} (414), depois o campo cerrado (384) e por último o campo sujo (306) (Figura \ref{fig:Figura 2}).

Com relação às três fitofisionomias amostradas, o maior número de registros foi realizado na fitofisionomia campo cerrado (653), seguido por campo sujo (640) e cerrado \textit{sensu stricto} (625). Entretanto, quando consideramos o número de registros realizados dentro ou fora do raio amostral de 100 m em torno dos pontos amostrais, os resultados são qualitativamente diferentes.
Os registros dentro do raio amostral de 100m foram maiores na fitofisionomia campo sujo (334), depois em campo cerrado (269) e por último em cerrado \textit{sensu stricto} (Figura \ref{fig:Figura 2}) (214). Já para os registros realizados fora do raio de 100 m torno dos pontos amostrais, a fitofisionomia que apresentou maior número de registros foi o cerrado \textit{sensu stricto} (414), depois o campo cerrado (384) e por último o campo sujo (306) (Figura \ref{fig:Figura 2}).
