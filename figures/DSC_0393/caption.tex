\textbf{Figura cauda}: Parte ventral da nadadeira caudal de uma baleia-jubarte foto-identificada  no Banco dos Abrolhos em 2014.

Análises de uma série temporal de 11 anos de levantamentos aéreos e 22 anos de dados de identificação individual das baleias-jubarte que reproduzem no Brasil foram realizadas visando estimar as taxas de crescimento dessa população. Foram utilizados métodos que consideram a detecção imperfeita para as análises. Para os levantamentos aéreos foram construídas funções de detecção anuais com os dados de amostragem de distâncias perpendiculares dos grupos com relação às linhas de transecção. Essas funções de detecção descrevem como a probabilidade de detecção de grupos de baleias decresce conforme a distância da linha. Calcula-se então a densidade de baleias considerando a probabilidade de detecção em uma faixa determinada de distância da linha de transecção, resultando em estimativas mais robustas de abundância. Os dados de foto-identificação foram analisados utilizados modelos de marcação-recaptura, que permitem o cálculo de probabilidades de captura para os indivíduos marcados da população, resultando em estimativas de sobrevivência e taxa de crescimento mais robustas. Essas análises estão descritas no manuscrito que se encontra anexo a este relatório. O manuscrito está em fase final de revisão pelos autores e será submetido durante a vigência deste projeto de auxílio.