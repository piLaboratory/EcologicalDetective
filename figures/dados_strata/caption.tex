\textbf{Figura 4}: Número de registros e de espécies observados em cada um dos estratos vegetacionais definidos, considerando apenas os registros feitos dentro do raio amostral de 100m. A definição dos estratos vegetacionais encontra-se no corpo do texto. Note que o comprimento do eixo y é diferente nos dois gráficos.

Estes resultados mostram que o número de espécies registradas nos estratos mais baixos é menor que o número de espécies em registros mais altos, más isto não nos possibilita concluir que a utilização destes estratos é maior, visto que estes estratos são também mais visíveis e podem possibilitar uma detecção maior. Esta proposição parece ter suporte pelos registros fora do raio amostral de 100m em torno dos pontos amostrais, onde somente sete registros foram realizados, sendo um registro de uma espécie no estrato 3 e seis registros de seis espécies diferentes no estrato 4. Assim, quanto maior a distância dos registros, mais dificultada parece ser a detecção de registros em estratos mais baixos. Ao final da última campanha de campo e das análises finais dos dados iremos analisar os padrões aqui apresentados considerando os efeitos da detectabilidade. Com isto, esperamos gerar subsídios para alcançar uma maior compreensão sobre a influência da complexidade dos habitats sobre a diversidade de aves, assim como também um melhor entendimento dos fatores que podem influenciar a detecção destes padrões.