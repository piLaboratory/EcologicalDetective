\textbf{Figura Moria}: Embarcação utilizada para a amostragem de baleias-jubarte no Banco dos Abrolhos em 2014.

Entre julho e novembro de 2014 foram realizadas campanhas marítimas embarcadas para contagem e foto-identificação de baleias-jubarte no Banco dos Abrolhos. Considerando o ano de 2014, foram completados 25 anos de série temporal de dados disponíveis para as análises de taxas demográficas e dinâmica populacional dessa espécie em sua área reprodutiva no Brasil. A embarcação utilizada foi o \textit{trawler} Moriá, com casco de madeira medindo cerca de 15 metros de comprimento (Figura Moria).  \textbf{Figura Moria}: Embarcação utilizada para a amostragem de baleias-jubarte no Banco dos Abrolhos em 2014.   Foram realizadas dez campanhas de campo, totalizando 28 dias de amostragem ao longo de quase 1.150 milhas náuticas percorridas em esforço (Tabela amostragem).  

\textbf{Tabela amostragem}: Esforço amostral realizado durante a temporada reprodutiva da baleia-jubarte no Banco dos Abrolhos em 2014.  

\begin{tabular}{lcccc}  
\textbf{Campanha} & \textbf{Datas} & \textbf{Dias amostrados} & \textbf{Milhas náuticas percorridas} & \textbf{Grupos de baleia-jubarte} \\
01 & 02 a 04 de agosto & 3 & 129,0 & -- \\
02 & 21 a 23 de agosto & 3 & 126,6 & -- \\
03 & 27 a 29 de agosto & 3 & 115,3 & -- \\
04 & 11 a 13 de agosto & 3 & 122,1 & -- \\
05 & 24 a 26 de setembro & 3 & 138,1 & -- \\
06 & 08 a 10 de outubro & 3 & 109,2 & -- \\
07 & 24 a 26 de outubro & 3 & 93,1 & -- \\
08 & 01 a 04 de novembro & 4 & 209,7 & -- \\
09 & 07 e 08 de novembro & 2 & 66,5 & -- \\
10 & 12 de novembro & 1 & 39,2 & -- \\
textbf{TOTAL} & & \textbf{28} & \textbf{1.149,9} & \\
     
\end{tabular}    

To be continued soon... 