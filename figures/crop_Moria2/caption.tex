\textbf{Figura Moria}: Embarcação \textit{Moriá}, utilizada para a amostragem de baleias-jubarte no Banco dos Abrolhos em 2014.

Foram realizadas dez campanhas de campo, totalizando 28 dias de amostragem ao longo de quase 1.150 milhas náuticas percorridas em esforço (Tabela amostragem). Um total de 273 baleias-jubarte foi identificada através de marcas naturais na parte ventral da nadadeira caudal (Figura cauda). Uma análise preliminar revelou pelo menos nove reavistagens de baleias identificadas em anos anteriores, mas esse número deve aumentar substancialmente após a realização da comparação do catálogo, que está em andamento. Amostras de pele para extração de DNA micro-satélites, que também permite a identificação individual, foram coletadas para 91 indivíduos.

\textbf{Tabela amostragem}: Esforço amostral realizado durante a temporada reprodutiva da baleia-jubarte no Banco dos Abrolhos em 2014.  

\begin{tabular}{lcccc}  
\textbf{Campanha} & \textbf{Datas} & \textbf{Dias amostrados} & \textbf{Milhas náuticas percorridas} & \textbf{Grupos de baleia-jubarte} \\
01 & 02 a 04 de agosto & 3 & 129,0 & -- \\
02 & 21 a 23 de agosto & 3 & 126,6 & -- \\
03 & 27 a 29 de agosto & 3 & 115,3 & -- \\
04 & 11 a 13 de agosto & 3 & 122,1 & -- \\
05 & 24 a 26 de setembro & 3 & 138,1 & -- \\
06 & 08 a 10 de outubro & 3 & 109,2 & -- \\
07 & 24 a 26 de outubro & 3 & 93,1 & -- \\
08 & 01 a 04 de novembro & 4 & 209,7 & -- \\
09 & 07 e 08 de novembro & 2 & 66,5 & -- \\
10 & 12 de novembro & 1 & 39,2 & -- \\
\textbf{TOTAL} & & \textbf{28} & \textbf{1.149,9} & \\
\end{tabular}    
