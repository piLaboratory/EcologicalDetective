\textbf{Figura 3}: Número de espécies registradas em cada uma das fitofisionomias amostradas, considerando os rgistros feitos dentro do raio amostral de 100m e fora deste raio amostral. Legenda: CS- Campo sujo; CC -  Campo Cerrado; SS- Cerrado \textit{Sensu stricto}. Note que o comprimento do eixo y é diferente nos dois gráficos.

Estes resultados preliminares indicam que o número de registros feitos em cada fitofisionomia, assim como o número de espécies presentes em cada uma, são diferentes. Além disso, estes resultados indicam que existe um efeito da distância que os registros foram realizados, já que os registros e espécies encontrados dentro e fora do raio de 100m em torno dos pontos apresentam diferentes padrões. Este efeito da distância pode ocorrer tanto por uma maior heterogeneidade de hábitats em torno de algumas fitofisionomias em relação à outras quanto por um efeito da detectabilidade associado às características dos hábitats, já que o efeito positivo da vegetação sobre o número de espécies poderia ser reduzido ou até mascarado por um efeito negativo da vegetação sobre a detecção das espécies. Ao final da coleta e das análises dos dados, esperamos obter uma maior compreensão sobre a influência da complexidade dos habitats sobre a diversidade de aves, assim como também um melhor entendimento dos fatores que podem influenciar a detecção destes padrões. 

O uso dos estratos pelas aves também foi analisado de maneira preliminar e dentro do raio de 100m em torno dos pontos foram realizados 15 registros no estrato 1, 72 no estrato 2, 116 no estrato 3 e 313 no estrato 4 (Figura \ref{fig:Figura 4}). Com relação ao número de espécies dentro deste mesmo raio amostral, foram observadas 12 espécies no estrato 1, 28 espécies no estrato 2, 34 no estrato 3 e 63 no estrato 4 (Figura \ref{fig:Figura 4}).