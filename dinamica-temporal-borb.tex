\subsection{2.1.1 Dinâmica temporal} %Kiwi
\label{sec:dinamica-temporal-borb}
A distribuição temporal das capturas de indivíduos de Ithomiini não foi homogênea entre os meses. Foram observadas grandes concentrações de indivíduos nos meses mais secos (maio a agosto) e números de indivíduos consideravelmente mais baixos nos meses chuvosos (ver figura 2.1-3). Estas agregações sazonais de Ithomiini – sempre durante a estação menos chuvosa - já foram observadas em regiões onde ocorre uma estação relativamente mais seca ao longo do ano, como em Campinas (Brown Jr. and Freitas 2002) e Caldas Novas (GO - Pinheiro et al. 2008). 

Figura
Figura 2.1-3. Distribuição temporal das taxas de captura (número de indivíduos/tempo de amostragem (min) * número de coletores) dos Ithomiini (Nymphalidae, Danainae) amostrados no Parque Esporte para Todos (PEPT) e na Reserva Florestal da Cidade Universitária Armando Salles Oliveira.

As abundâncias – representadas pelo número bruto de capturas - das espécies de Ithomiini capturadas no Parque Esporte Para Todos (PEPT) variaram consideravelmente entre os meses. Espécies inicialmente raras (ou ausentes) aumentaram em número em meses posteriores e vice-versa, evidenciando a existência de rotatividade (turnover) considerável nas Distribuições de Abundâncias de Espécies (DAE) obtidas em cada mês (figura 2.1-4).  

Figura
Figura 2.1-4. Distribuições de Abundância de Espécies (DAE) de Ithomiini obtidas em cada mês (abr/13 a out/14) no Parque Esporte Para Todos (PEPT). No eixo y está representado o número de indivíduos capturados e no eixo x estão representadas as 22 espécies registradas na área. A ordem das espécies (rank) é mantida constante ao longo dos meses.

Para mensurar a rotatividade (turnover) na composição e abundância da comunidade de Ithomiini do Parque Esporte Para Todos nós calculamos a similaridade (Bray-Curtis) das amostras de cada mês em relação ao centróide da comunidade, o qual foi obtido tirando-se a média do número de indivíduos de cada espécie capturados em todos os meses de amostragem. Os resultados revelam uma tendência de queda na similaridade da comunidade em relação ao seu centróide na estação chuvosa e um posterior aumento na estação seca (figura 2.1-5).

Figura
Figura 2.1-5. Similaridade de Bray-Curtis das amostras mensais de Ithomiini com o centroide da comunidade, o qual foi obtido calculando-se o número médio de indivíduos de cada espécie capturados por mês.

A curva-espécies tempo baseada em indivíduos obtida no Parque Esporte para Todos de abril/13 a out/14 está representada na figura 2.1-6. É possível notar uma tendência à estabilização da curva indicando que as espécies mais comuns da comunidade foram registradas e que novas espécies que eventualmente podem ser capturadas devem ser representadas por indivíduos vagantes.

Figura
Figura 2.1-6. Curva espécies-tempo baseada em indivíduos obtida no Parque Esporte Para Todos entre abril/13 e outubro/14. OS dados foram aleatorizados 100 vezes e as linhas mais estreitas representam o intervalo de confiança de 95%.

Nas próximas etapas do projeto os dados empíricos (DAE, turnover e curvas-espécies tempo) serão comparados com dados simulados sob um cenário de neutralidade. Ainda, serão explorados parâmetros demográficos das diferentes espécies (obtidos pela técnica de captura-marcação-recaptura) a fim de testar a premissa de equivalência demográfica evocada pela Teoria Neutra. Também serão testadas hipóteses a respeito de equivalência demográfica dentro – porém não entre – anéis miméticos.