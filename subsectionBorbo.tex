\subsection{2.1 Borboletas: dinâmica de comunidades e neutralidade} %Kiwi
\label{sec:dinamica-temporal-borb} 
Até o momento estão sendo realizadas amostragens regulares de captura-marcação-recaptura de borboletas Ithomiini (Nymphalidae, Danainae) em duas áreas da Cidade Universitária em São Paulo (USP). Na primeira, o Parque Esporte Para Todos (PEPT), as amostragens iniciaram-se em de abril de 2013 e até dezembro de 2014 haviam sido realizadas 24 campanhas. A frequência de coletas foi aproximadamente quinzenal de abril a julho de 2013 e aproximadamente mensal após este período (ver tabela 2.1-1). As coletas da segunda área - Reserva Florestal da Cidade Universitária - iniciaram-se em fevereiro de 2014 e até dezembro de 2014 haviam sido realizadas 11 campanhas. Planeja-se continuar as coletas em ambas as áreas pelo menos até dezembro de 2015 com frequência mensal. Aqui serão reportados os dados obtidos até outubro de 2014.

Tabela 2.1-1 Local e data das amostragens de captura-marcação-recaptura de borboletas Ithomiinae (Nymphalidae, Danainae) na Cidade Universitária Armando de Salles Oliveira (USP). PEPT – Parque Esporte Para Todos.

Ao longo das amostragens realizadas entre abril/13 e outubro/14 foram capturados 9.300 indivíduos de Ithomiini: 7.323 no PEPT e 1.977 na Reserva Florestal. Ao todo foram registradas 23 espécies divididas em nove subtribos, representando sete anéis miméticos diferentes (tabela 2.1-2 e figura 2.1-1). 

Tabela 2.1-2.Relação das 23 espécies de borboletas da tribo Ithomiini (Lepidoptera, Nymphalydae, Danainae) capturadas no Parque Esporte Para Todos e na Reserva Florestal, ambas no campus da USP. Taxonomia e nomenclatura seguem proposta de Brower et al. (2014). Anel mimético definido de acordo com Willmott & Mallet (1993) - asteriscos representam espécies não citadas por Willmott & Mallet mas que devido à grande semelhança morfológica devem se encaixar nos anéis miméticos indicados. Core representa as espécies que tiveram número de capturas-recapturas suficiente para que parâmetros demográficos possam ser estimados.
Tabela





