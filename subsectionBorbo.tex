\subsection{Borboletas: dinâmica de comunidades e neutralidade} %Kiwi
\label{sec:dinamica-temporal-borb} 
Até o momento estão sendo realizadas amostragens regulares de captura-marcação-recaptura de borboletas Ithomiini (Nymphalidae, Danainae) em duas áreas da Cidade Universitária em São Paulo (USP). Na primeira, o Parque Esporte Para Todos (PEPT), as amostragens iniciaram-se em de abril de 2013 e até dezembro de 2014 haviam sido realizadas 24 campanhas. A frequência de coletas foi aproximadamente quinzenal de abril a julho de 2013 e aproximadamente mensal após este período (ver tabela 2.1-1). As coletas da segunda área - Reserva Florestal da Cidade Universitária - iniciaram-se em fevereiro de 2014 e até dezembro de 2014 haviam sido realizadas 11 campanhas. Planeja-se continuar as coletas em ambas as áreas pelo menos até dezembro de 2015 com frequência mensal. Aqui serão reportados os dados obtidos até outubro de 2014.

\textbf{Tabela 2.1-1}: Local e data das amostragens de captura-marcação-recaptura de borboletas Ithomiinae (Nymphalidae, Danainae) na Cidade Universitária Armando de Salles Oliveira (USP). PEPT – Parque Esporte Para Todos.
\begin[tabular}{lccc}
1 & PEPT & 1& 18/4/2013 \\
 &PEPT&2&18/4/2013 \\
 &PEPT&3&19/4/2013 \\
 &PEPT&4&19/4/2013 \\
2&PEPT&1&25/4/2013 \\
&PEPT&2&25/4/2013 \\
&PEPT&3&26/4/2013 \\
&PEPT&4&26/4/2013 \\
3&PEPT&1&2/5/2013 \\
&PEPT&2&2/5/2013 \\
&PEPT&3&3/5/2013 \\
&PEPT&4&4/5/2013 \\
4&PEPT&1&9/5/2013 \\
&PEPT&2&9/5/2013 \\
&PEPT&3&10/5/2013 \\
&PEPT&4&11/5/2013 \\
5&PEPT&1&13/6/2013 \\
&PEPT&2&13/6/2013 \\
&PEPT&3&14/6/2013 \\
&PEPT&4&15/6/2013 \\
6&PEPT&1&29/6/2013 \\
&PEPT&2&29/6/2013 \\
7&PEPT&1&4/7/2013 \\
&PEPT&2&4/7/2013 \\
&PEPT&3&5/7/2013 \\
&PEPT&4&6/7/2013 \\
8&PEPT&1&18/7/2013 \\
&PEPT&2&18/7/2013 \\
&PEPT&3&19/7/2013 \\
&PEPT&4&20/7/2013 \\
9&PEPT&1&8/8/2013 \\
&PEPT&2&8/8/2013 \\
&PEPT&3&9/8/2013 \\
&PEPT&4&10/8/2013 \\
10&PEPT&1&20/9/2013 \\
&PEPT&2&20/9/2013 \\
&PEPT&3&21/9/2013 \\
&PEPT&4&21/9/2013 \\
11&PEPT&1&17/10/2013 \\
&PEPT&2&18/10/2013 \\
&PEPT&3&18/10/2013 \\
&PEPT&4&19/10/2013 \\
12&PEPT&1&14/11/2013 \\
&PEPT&2&14/11/2013 \\
&PEPT&3&16/11/2013 \\
13&PEPT&1&12/12/2013 \\
&PEPT&2&12/12/2013 \\
&PEPT&3&13/12/2013 \\
&PEPT&4&13/12/2013 \\
14&PEPT&1&27/1/2014 \\
&PEPT&2&27/1/2014 \\
&PEPT&3&28/1/2014 \\
&PEPT&4&28/1/2014 \\
15&PEPT&1&17/2/2014 \\
&PEPT&2&17/2/2014 \\
&PEPT&3&18/2/2014 \\
&PEPT&4&18/2/2014 \\
&Reserva Florestal&1&17/2/2014 \\
&Reserva Florestal&2&17/2/2014 \\
&Reserva Florestal&3&18/2/2014 \\
&Reserva Florestal&4&18/2/2014 \\
16&Reserva Florestal&1&17/3/2014 \\
&Reserva Florestal&2&17/3/2014 \\
&Reserva Florestal&3&18/3/2014 \\
&Reserva Florestal&4&18/3/2014 \\
&PEPT&1&17/3/2014 \\
&PEPT&2&17/3/2014 \\
&PEPT&3&18/3/2014 \\
&PEPT&4&18/3/2014 \\
17&PEPT&1&14/4/2014 \\
&PEPT&2&14/4/2014 \\
&Pept&3&15/4/2014 \\
&PEPT&4&16/4/2014 \\
&Reserva Florestal&1&14/4/2014 \\
&Reserva Florestal&2&14/4/2014
&Reserva Florestal&3&15/4/2014
&Reserva Florestal&4&16/4/2014
18&PEPT&1&12/5/2014
&PEPT&2&12/5/2014
&PEPT&3&13/5/2014
&PEPT&4&13/5/2014
&Reserva Florestal&1&12/5/2014
&Reserva Florestal&2&12/5/2014
&Reserva Florestal&3&13/5/2014
&Reserva Florestal&4&13/5/2014
19&Reserva Florestal&1&23/6/2014
&Reserva Florestal&2&23/6/2014
&Reserva Florestal&3&24/6/2014
&Reserva Florestal&4&24/6/2014
&PEPT&1&23/6/2014
&PEPT&2&24/6/2014
&PEPT&3&24/6/2014
&PEPT&4&25/6/2014
20&PEPT&1&22/7/2014
&PEPT&2&22/7/2014
&PEPT&3&23/7/2014
&PEPT&4&23/7/2014
&Reserva Florestal&1&22/7/2014
&Reserva Florestal&2&22/7/2014
&Reserva Florestal&3&23/7/2014
&Reserva Florestal&4&23/7/2014
21&PEPT&1&18/8/2014
&PEPT&2&18/8/2014
&PEPT&3&19/8/2014
&PEPT&4&19/8/2014
&Reserva Florestal&1&18/8/2014
&Reserva Florestal&2&18/8/2014
&Reserva Florestal&3&19/8/2014
&Reserva Florestal&4&19/8/2014
22&PEPT&1&25/9/2014
&PEPT&2&25/9/2014
&PEPT&3&26/9/2014
&Reserva Florestal&1&25/9/2014
&Reserva Florestal&2&25/9/2014
&Reserva Florestal&3&26/9/2014
23&PEPT&1&23/10/2014
&PEPT&2&23/10/2014
&PEPT&3&24/10/2014
&PEPT&4&24/10/2014
&Reserva Florestal&1&23/10/2014
&Reserva Florestal&2&23/10/2014
&Reserva Florestal&3&24/10/2014
&Reserva Florestal&4&24/10/2014
24&Reserva Florestal&1&24/11/2014
 &Reserva Florestal&2&24/11/2014
 &Reserva Florestal&3&25/11/2014
 &Reserva Florestal&4&25/11/2014
 &PEPT&1&24/11/2014
 &PEPT&2&24/11/2014
 &PEPT&3&25/11/2014
 &PEPT&4&25/11/2014
\end{tabular}
Tabela 2.1-1

Ao longo das amostragens realizadas entre abril/13 e outubro/14 foram capturados 9.300 indivíduos de Ithomiini: 7.323 no PEPT e 1.977 na Reserva Florestal. Ao todo foram registradas 23 espécies divididas em nove subtribos, representando sete anéis miméticos diferentes (tabela 2.1-2 e figura 2.1-1). 

Tabela 2.1-2






