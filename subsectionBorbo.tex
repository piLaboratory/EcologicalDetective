\subsection{2.1 Borboletas: dinâmica de comunidades e neutralidade} %Kiwi
\label{sec:dinamica-temporal-borb} 
Até o momento estão sendo realizadas amostragens regulares de
captura-marcação-recaptura de borboletas Ithomiini (Nymphalidae,
Danainae, Fig.~\ref{fig:2.1.1}) 
em duas áreas da Cidade Universitária em São Paulo (USP). Na
primeira, o Parque Esporte Para Todos (PEPT), as amostragens
iniciaram-se em de abril de 2013 e até dezembro de 2014 haviam sido
realizadas 24 campanhas. A frequência de coletas foi aproximadamente
quinzenal de abril a julho de 2013 e aproximadamente mensal após este
período (ver tabela 2.1-1). As coletas da segunda área - Reserva
Florestal da Cidade Universitária - iniciaram-se em fevereiro de 2014
e até dezembro de 2014 haviam sido realizadas 11 campanhas. Planeja-se
continuar as coletas em ambas as áreas pelo menos até dezembro de 2015
com frequência mensal. Aqui serão reportados os dados obtidos até
dezembro de 2014, porém algumas análises incluem dados até outubro de 2014.

Como observado em qualquer comunidade biológica, o número de
indivíduos capturados em ambas as áreas não foi homogeneamente
distribuído entre as espécies. A maioria dos indivíduos pertenceu a
poucas espécies, sendo que das 23 espécies registradas apenas oito
podem ser consideradas espécies \emph{core} e apresentam número de capturas
suficiente para que parâmetros demográficos possam ser calculados
(Fig.~\ref{fig:2.1.2}). 

\textbf{Tabela 2.1-1} Local e data das amostragens de
captura-marcação-recaptura de borboletas Ithomiini (Nymphalidae,
Danainae) na Cidade Universitária Armando de Salles Oliveira
(USP). PEPT – Parque Esporte Para Todos.

Ao longo das amostragens realizadas entre abril/13 e dezembro/14 foram
capturados 7.758 indivíduos de Ithomiini (mais 2784 recapturas) : 5.939 no PEPT (mais 2.309) recapturas) e 1.864 na Reserva Florestal (mais 520 recapturas). Ao todo foram registradas 23 espécies divididas em nove subtribos, representando sete anéis miméticos diferentes (Tab.~\ref{tab:borb1} e Fig.~\ref{fig:2.1-2}).

\begin{table}
\caption{\label{tab:borb1} Número de capturas e recapturas de indivíduos 
de borboletas da tribo
Ithomiini (Lepidoptera, Nymphalydae, Danainae) no Parque
Esporte Para Todos (PEP) e na Reserva Florestal (RF), ambas no campus da
USP. Taxonomia e nomenclatura seguem proposta de Brower et
al. (2014). Anel mimético definido de acordo com Willmott \& Mallet
(1993)}
\begin{tabular}{llrrrr}
  \textbf{Taxon} & \textbf{Anel} & \textbf{Cap-PEP} & \textbf{Recap-PEP} & \textbf{Cap-RF} & \textbf{Recap-RF}\\
  Melinaea ludovica paraiya & Ethra & 1 & 0 & 0 & 0\\
  Methona themisto & Themisto & 10 & 0 & 1 & 0\\
  Aeria olena & Agna & 5 & 3 & 0 & 0\\
  Mechanitis lysimnia & Lysimnia & 230 & 82 & 87 & 33\\
  Mechanitis polymnia & Ethra & 771 & 238 & 182 & 36\\
  Dircenna dero & Themisto & 41 & 3 & 19 & 0\\
  Episcada carcinia & Aquata & 4 & 0 & 7 & 0\\
  Episcada clausina & Aquata & 10 & 1 & 9 & 0\\
  Episcada hymenaea & Hymenaea & 45 & 5 & 32 & 0\\
  Episcada philoclea & Philoclea & 0 & 0 & 1 & 0\\
  Ithomia agnosia zikani & Aquata & 1999 & 435 & 546 & 65\\
  Ithomia drymo & Aquata & 16 & 2 & 9 & 1\\
  Placidina euryanassa & Lysimnia & 2 & 0 & 4 & 0\\
  Epityches eupompe & Philoclea & 285 & 63 & 74 & 5\\
  Hypothyris euclea laphria & Ethra & 448 & 189 & 224 & 117\\
  Hypothyris ninonia daeta & Lysimnia & 1539 & 1164 & 359 & 204\\
  Oleria aquata & Aquata & 8 & 1 & 11 & 5\\
  Hypoleria lavinia & Aquata & 212 & 65 & 193 & 49\\
  Heterosais edessa & Aquata & 10 & 0 & 0 & 0\\
  Mcclungia cymo salonina & Philoclea & 248 & 48 & 58 & 4\\
  Pseudoscada acilla & Aquata & 29 & 6 & 18 & 0\\
  Pseudoscada erruca & Aquata & 21 & 3 & 10 & 1\\
  TOTAL & & 5934 & 2308 & 1844 & 520\\
\end{tabular} 
\end{table}