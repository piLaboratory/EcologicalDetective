\documentclass[12pt, A4]{article}
\usepackage[english,brazil]{babel}
\usepackage[utf8]{inputenc}
\usepackage[T1]{fontenc}
\usepackage{natbib}
\usepackage{url}
\usepackage{graphicx}
\usepackage{fancyhdr,multicol}
\usepackage[paper=a4paper, textwidth=5.5in, top=1in, bottom=1.25in, headheight=.5in, headsep=0.5in, footskip=0.5in]{geometry}
% \setlength{\hoffset}{-0.5in}
% \setlength{\textwidth}{6in}
% \setlength{\voffset}{-0.1in}
% \setlength{\textheight}{10in}
% \setlength{\parindent}{0pt}
\pdfpagewidth=\paperwidth
\pdfpageheight=\paperheight
\setlength{\parskip}{1ex plus 0.5ex minus 0.2ex} % espaco entre paragrafos e retira tabulacao no inicio
\newcommand{\R}{{\sf R}}
\newcommand{\code}[1]{\texttt{#1}}
\renewcommand{\footrulewidth}{0.5pt}
\begin{document}
%\maketitle
\pagestyle{fancyplain}
\lhead[] {\textsl{\footnotesize Paulo Inácio K.L. Prado} \\ \textsl{\footnotesize Projeto de Bolsa Produtividade em Pesquisa -- 2014}}
\rhead[\fancyplain{}{\textsl{\thepage}}]%
      {\fancyplain{}{\textsl{\thepage}}}
\rfoot[]{\textsl{\footnotesize Laboratório de Ecologia Teórica -- IBUSP} \\
  \footnotesize \url{http://ecologia.ib.usp.br/let}}
\lfoot[]{}
\cfoot[]{}
\thispagestyle{empty}
%%%%%%%%%%%%%%%%%%%%%%%%%%%%%%%%%%%%%%%%%%%%%%%%%%%%%%%%%%%%%%%%%%%%%%%%%%%%%%%%%%%%%%%
%\vspace{-10in}
\begin{center}
	{\large Projeto de Pesquisa}	\\\vspace{0.5\baselineskip}
	\textbf{\Large Ferramentas do Detetive Ecológico: \\\vspace{.25\baselineskip}
          uso e avaliação de modelos com detecção imperfeita} \\\vspace{1.5\baselineskip}
	{\normalsize Paulo Inácio de Knegt López de Prado } \\\vspace{.5\baselineskip}
        {\normalsize Depto. de Ecologia, Instituto de Biociências, Universidade de São Paulo}	%\\\vspace{2\baselineskip}
\end{center}


\begin{abstract}
O desenvolvimento e aplicação de teoria ecológica depende de estimativas
não-enviesadas e precisas da ocorrência, tamanho e taxas vitais de
populações naturais. Há uma percepção crescente de que
tais estimativas são fortemente afetadas por erros de observação, dos
quais as falsas ausências são o exemplo mais óbvio. Assim, o
pressuposto de detecção perfeita ou invariável de espécies e
indivíduos em sido fortemente questionado, bem como os resultados que
dele derivam. Em resposta a esse problema desenvolveu-se um formalismo
de modelos estatísticos que separa dois níveis de
variação dos dados: a camada de observação e a camada do processo
biológico. Os  modelos hierárquicos com detecção
imperfeita são resultado desse formalismo. Têm implementações
computacionais acessíveis a ecólogos, e podem ser aplicadas em muito
mais casos do que têm sido. Por outro lado, essa nova abordagem
envolve procedimentos estatísticos mais complexos e um maior esforço
de amostragem. Assim, é necessário avaliar os ganhos efetivos dos
modelos com detecção imperfeita em diferentes tipos de pesquisas
ecológicas. O objetivo desta proposta é usar estimativas de
ocorrência, abundância e taxas vitais de espécies obtidos com modelos
de detecção imperfeita em três estudos de casos que juntos abrangem
testes sobre estrutura e dinâmicas de populações e comunidades, em
aspectos teóricos e aplicados. Além da importância teórica e aplicada
dos estudos de caso em si, vamos avaliar a sensibilidade dos
resultados e conclusões à adoção da abordagem de modelos com detecção
imperfeita.
\end{abstract}

\selectlanguage{english}
  \begin{abstract}
The advancement and application of ecological theory depends on
unbiased and precise estimates of occurrence, size and vital rates of
populations in ecological systems. There is a growing concern that
such estimates are strongly affected by observation bias; the most
obvious case are false absences. Thus, the assumption of perfect or
constant detection of species and individuals has been strongly
challenged, and hence the results derived from this assumption. A
formalism of hierarchical statistical models to tackle the problem is
now available, as well as computational tools to fit these
models. Such formalism separates two levels of variation of data: the
observation layer and the layer of the biological
processes. Colletively known  as hierarchical models with imperfect
detection, such statistical methods could be applied much more widely
than they currently are. Nevertheless, this new approach involves
complex statistical procedures and larger sampling efforts. Thus, it
is necessary to evaluate the effective gains of the models with
imperfect detection in different types of ecological research. The
goal of this proposal is to use estimates of occurrence, abundance and
vital rates of species obtained with imperfect detection models in
three case studies. We selected cases that encompass a wide range of
tests on the structure and dynamics of populations and communities in
theoretical and applied situations. Besides the importance of the
theoretical and applied case studies in themselves, we aim also to
evaluate the sensitivity of the results and conclusions to the
inclusion of imperfect detection.
  \end{abstract}
\selectlanguage{brazil}

\section{Contextualização Geral}
\label{sec:cont-geral}

A ecologia é reconhecida como a análise dos fatores que determinam a
distribuição e abundância dos seres vivos, que são expressas
operacionalmente pelos registros de ocorrência e pela contagem de
indivíduos em campo. Assim, boa parte da pesquisa em ecologia manteve
o pressuposto de que número observado de indivíduos em sítios de
amostragem são
estimadores não enviesados, ou pelo menos bons índices, da abundância
e distribuição dos indivíduos. Mas, e se este pressuposto não for
verdadeiro para uma grande parte de estudos? Será que as bases da
ecologia estariam fundadas sob um terreno instável?

\subsection*{Detecção imperfeita}
\label{sec:deteccao-imperfeita}

Para qualquer um que se ocupe da abundância e distribuição de espécies
é óbvio que a detecção é imperfeita. Mesmo  para plantas, que são
sésseis e cujos problemas de detectabilidade são menos comuns, a
detecção imperfeita pode ser um problema para as estimativas, como
quando não estão aparentes devido à dormência
\citep[e.g.][]{shefferson2007}. 
No entanto, apenas nas últimas décadas houve na ecologia
uma formalização estatística da detecção, que levou ao
desenvolvimento de modelos preditivos que a considerem. A detecção de
uma espécie em campo, por exemplo, é indicativa de sua presença, mas a
não detecção não significa necessariamente que a espécie está ausente
\citep{mackenzie2002}. Falsas ausências são o caso mais básico da
detecção imperfeita, e fazem com que a proporção de
áreas onde uma determinada espécie ocorre seja subestimada
\citep{mackenzie2002}. 
Outros problemas de detecção ainda podem ser comuns e afetam a
maioria das amostragens de populações, como duplas contagens, falsos
positivos e erros de classificação \citep{buckland2001introduction}
A situação
mais comum em estudos que amostram populações de animais é a de que a
detectabilidade varia no tempo e no espaço em decorrência de fatores
ambientais e ecológicos \citep{williams2002}.
Esses potenciais fatores de confusão podem
e afetar muito a precisão e acurácia das estimativas e, portanto,
as conclusões de um estudo \citep{mackenzie2006}. 

Podemos estender ainda o conceito da detecção imperfeita aos modelos
de marcação-recaptura de indivíduos em uma população pois, novamente,
nem todos os indivíduos são detectados. Neste caso, tanto a estimativa
de abundância de uma população quanto as taxas de recrutamento e
sobrevivência dos indivíduos (dois componentes importantes que
governam o crescimento de uma população) podem ser subestimados se a
detecção imperfeita dos indivíduos não for considerada 
\citep{gimenez2008}. Estimativas de crescimento de uma população podem diferir
substancialmente entre modelos que consideram a detectabilidade e
modelos que ignoram este problema \citep{clark2004,kery2009}.
Em estudos de comunidades, tanto a riqueza pode ser
subestimada como as taxas de renovação (\emph{turnover}) podem ser
superestimadas se não considerarmos a detectabilidade 
\citep{mackenzie2003,dorazio2005}. 
Portanto, a resposta à pergunta
deste projeto de pesquisa é sim, 
o pressuposto de detecção perfeita está errado.

A questão em aberto é o quanto os vieses e imprecisões decorrentes afetaram - e
afetam - a teoria ecológica e sua aplicação. Não temos uma resposta,
pois o uso de modelos estatísticos com detecção imperfeita ainda é
recente e pouco disseminado 
\citep{williams2002}. O uso ainda restrita deve-se em parte à
dificuldade de aplicar esses modelos em muitas situações, e a dúvidas
sobre as eventuais vantagens
\citep{RD2008,banks2014}. 
Ainda assim, é provável que a utilização de
modelos que consideram a detecção imperfeita não irá mudar
drasticamente o arcabouço da teoria ecológica. As leis e fundamentos
da ecologia em geral apoiam-se em teorias e modelos cujas
previsões gerais não dependem de valores específicos de parâmetros
\citep{scheinerbook2011}. Por outro lado, os testes empíricos
derivados de teorias gerais muitas vezes dependem de previsões
quantitativas precisas, bem como a maioria das aplicações. Portanto, o
problema da detecção imperfeita pode não atingir o cerne da teoria
ecológica, mas poderia afetar as inúmeras hipóteses periféricas e as
subsequentes generalizações que compõem a teoria ecológica, e também
sua utilidade prática. Testar essas hipóteses periféricas constitui a
maior parte do trabalho dos cientistas e da produção científica de uma
disciplina \citep{kuhn1970}.
Nesse contexto o conceito de detecção
imperfeita pode contribuir para a ecologia em dois aspectos
principais: (a) estimativas mais precisas e acuradas podem contribuir
para refinamentos teóricos, e também para estudos de caso e aplicações
específicas;  (b) o desenvolvimento de uma teoria de hierarquia que
explicite a separação e interação entre  processos ecológicos e
observacionais nos modelos estatísticos que usamos \citep{RD2008}. 
Assim, ignorar a questão da detectabilidade não nos parece uma
opção sensata considerando sua potencial contribuição teórica,
analítica e aplicada, e dado o conjunto hoje disponível de técnicas
que incorporam explicitamente a detecção imperfeita.

\subsection*{Modelos estatísticos em ecologia}
\label{sec:model-estat-em}

O trabalho do ``detetive ecológico'' é identificar a explicação mais
plausível para um fenômeno entre muitas alternativas, à luz das
evidências disponíveis, necessariamente incompletas. É uma abordagem
baseada em modelos cujo formalismo estatístico tem quatro princípios
básicos \citep{Hilborn1997,RD2008,kery2012}:
\begin{itemize}
\item Modelos estatísticos são formulações das previsões de uma dada
  hipótese sobre o comportamento de quantidades de interesse;
\item os modelos descrevem as quantidades como variáveis aleatórias, o
  que permite inferências mesmo sob a incerteza intrínseca que há em
  qualquer estimativa;
\item o comportamento de quantidades de interesse é resultados de duas
  fontes de variação: os processos ecológicos e o processo
  observacional ou de tomada de dados;
\item testes de hipóteses concorrentes se dão pela comparação
  simultânea dos modelos estatísticos delas derivados. 
\end{itemize}

Os modelos que consideram a detecção imperfeita que iremos usar seguem
a mesma lógica básica, que depende de um desenho amostral específico:
a replicação temporal da amostragem nas mesmas áreas, durante um
período de tempo onde assume-se que não há mudanças na ocupação ou
ocorrência. Tal desenho permite estimar uma probabilidade de detecção
da espécie, o que resolve o problema das falsas ausências
\citep{mackenzie2002}. 
A mesma ideia é aplicada para dados de
marcação e recaptura, quando usamos diferentes ocasiões de captura em um
curto intervalo de tempo para estimar a probabilidade de detecção dos
indivíduos. Tanto as estimativas de presença e abundância como as de
detecção podem ainda ser modeladas em função de diferentes covariáveis
ambientais, temporais ou mesmo de características da amostragem.
Assim, processo ecológicos e amostrais são explicitamente descritos em
submodelos distintos, que se combinam em uma hierarquia para descrever
o padrão encontrado \citep{RD2008}.

A proposta deste projeto é aplicar e avaliar o desempenho de
diferentes tipos de modelos que consideram a detecção imperfeita para
investigar questões ecológicas atuais em diferentes grupos
taxonômicos. Os sistemas e questões específicas foram escolhidos por
sua importância intrínseca para teoria e aplicação, pautados pela
experiência do responsável e sua equipe nas áreas de estrutura e dinâmica de
comunidades e de interações. Além disso, os temas e objetivos usam
métodos que representam bem a variedade de ferramentas estatísticas
de que os ecólogos dispõem para descrever as variações populacionais
animais com detecção imperfeita \citep[ver][]{williams2002}. Além dos
resultados de cada estudo de caso, esta proposta tem um quarto
objetivo metodológico e integrador, que é avaliar o impacto do uso dos
modelos propostos sobre testes de hipóteses em ecologia. Para isso, os
testes das hipóteses serão repetidos com estimativas que não
consideram a detecção imperfeita e as conclusões comparadas.  

Este projeto faz parte da minha linha principal de pesquisa, que é a
avaliação de protocolos de análise estatística inovadores em ecologia,
por meio de sua aplicação a problemas de relevância teórica e
aplicada. Assim, além dos aspectos metodológicos, esta proposta
pretende contribuir para questões em ecologia de comunidades e
conservação, os dois campos em que tenho concentrado minha atuação. A
motivação para avaliar métodos de estimativas com detecção imperfeita
é o seu potencial impacto em testes de muitas das hipóteses que temos
abordado, contempladas nos estudos de caso específicos descritos a
seguir. 

\section{Objetivos}
\label{sec:objetivos}

\subsection*{Geral}
\label{sec:geral}
Aplicar modelos com detecção imperfeita para testar hipóteses
sobre abundância e ocorrência de espécies e avaliar o desempenho
destes modelos, em relação à abordagem tradicional. 

\subsection*{Específicos}
\label{sec:especificos}

\begin{enumerate}
\item Testar se mudanças temporais na abundância relativa e
  diversidade de borboletas nectarívoras de subosque  
  são condizentes com previsões da Teoria Neutra ou com
  previsões de modelos baseados em nicho. 
\item Investigar se a correlações entre a diversidade de estratos da
  vegetação e diversidade de aves de cerrado é explicada pelos padrões
  de uso e sobreposição de uso desses estratos pelas aves.
\item Investigar os efeitos de flutuações climáticas a nível global na
  dinâmica populacional das baleias-jubarte que usam as águas
  brasileiras como área de reprodução.
\item Avaliar o impacto do uso das estimativas com detecção imperfeita
  sobre as conclusões obtidas no três estudos de caso acima.
\end{enumerate}

\section{Contextualização e métodos de cada objetivo}
\label{sec:cont-e-metod}

Este projeto esta dividido em quatro objetivos específicos, sendo que
os três primeiros envolvem o uso de dados empíricos que serão
coletados ou já foram coletados visando testar hipóteses ecológicas de
importância teórica e aplicada. O quarto objetivo envolve a integração
destes três bancos de dados para avaliar a eficácia dos modelos de
detectabilidade usados. A seguir descreveremos brevemente a abordagem
de construção e seleção de modelos que dará suporte à resposta das
três questões ecológicas que serão abordadas nos três primeiros
objetivos, o contexto de formulação de cada hipótese e os métodos
específicos que serão utilizados para atingir todos os objetivos. 

\subsection{Abordagem de construção e seleção de modelos}
\label{sec:abord-de-constr}

A lógica geral dos modelos que aplicaremos é usar um histórico de capturas dos
indivíduos ou de presença das espécies em sítios para estimar a
probabilidade de não tê-los observado em algumas campanhas/visitas,
mesmo que estejam presentes. Assim, distinguimos as falsas ausências
do desaparecimento real do indivíduo por morte ou emigração, e da
extinção ou emigração da espécie da área amostrada. A partir de um
histórico de detecções das espécies nos pontos amostrados em
diferentes visitas, ou dos indivíduos ao longo de ocasiões de captura,
é possível inferir os valores dos parâmetros de interesse (ocupação ou
sobrevivência) dos modelos por meio do método de máxima
verossimilhança. Os diferentes históricos de captura ou presença são
eventos mutuamente exclusivos que podemos descrever por funções de
probabilidade \citep{lebreton1992,mackenzie2002}. Com
isso definimos a função de verossimilhança do modelos, e com ela os
parâmetros de um modelo que maximizem a probabilidade deste modelo
descrever os dados obtidos \citep{williams2002}. De uma maneira
geral, usa-se uma distribuição de probabilidade para o processo
observacional (p.ex., probabilidade de detecção) condicionada a outra
distribuição para o processo ecológico de interesse (p.ex.,
probabilidade de sobrevivência ou de ocupação). Pode-se ainda tornar
os modelos bastante flexíveis se usarmos funções que expressam os
parâmetros das distribuições como combinações lineares de covariáveis
\citep{mccullagh1989}.

Esse método de inferência estatística também fornece uma medida de
plausibilidade do modelo, que reflete seu ajuste aos dados, e que pode
ser usado para comparação de diferentes modelos 
\citep{Johnson2004}. 
Com isto, o pesquisador pode formular modelos que representem
hipóteses ecológicas e de geração dos dados a serem testadas e avaliar
quais das hipóteses são mais plausíveis para explicar o conjunto
dos dados. Para a seleção do melhor modelo usaremos o Critério de
Informação de Akaike (AIC). Esse critério baseia-se no princípio da
parcimônia, ou seja, indica o modelo que melhor descreve o processo
que gerou os dados
usando o menor número de parâmetros \citep{Burnham2002}. Desta
forma, o melhor modelo dentre os concorrentes será a descrição mais
plausível e parcimoniosa da realidade ecológica de interesse, à luz da
evidência disponível. 

\subsection{Borboletas: dinâmica de comunidades e neutralidade}
\label{sec:dinamica-temporal-borb}

Um dos poucos padrões gerais em ecologia é a dominância numérica de
poucas espécies observada em qualquer comunidade biológica, a qual
imprime às distribuições de abundância de espécies (SADs, de 
\emph{species abudance distributions},) uma forma
côncava \citep{McGill2007,Prado2009}. 
Outros padrões gerais bem conhecidos são relação espécies-área
\citep{lomolino2000ecology} 
e a relação espécies-tempo (RET), ou seja, o incremento no número de
espécies registradas à medida que a área de estudo é ampliada e amostrada
repetidas vezes, respectivamente \citep{Preston1960,white2004,scheiner2011}.

Apesar de ser amplamente reconhecido que comunidades biológicas sofrem
mudanças ao longo do tempo 
há pouca informação sobre como e por que a diversidade destas
comunidades – que pode ser descrita pelas SADs - apresenta variações
temporais \citep{magurran2007}. 
Essa lacuna é
consequência da falta de séries temporais sistemáticas de comunidades
biológicas, e também consequência do fato de que modelos de abundância
de espécies e de variação de riqueza têm se focado muito mais em
padrões espaciais do que temporais
\citep{magurran2007,white2004,magurran2011}. 
Por isso ainda
é difícil saber o quanto mudanças temporais na diversidade de
comunidades biológicas são maiores do que mudanças esperadas por puro
acaso. 

A incorporação de uma perspectiva temporal no estudo das SADs pode
levar a um melhor entendimento dos padrões de diversidade biológica e
dos processos que os influenciam. Em estudos observacionais -
muitas vezes os únicos viáveis - as relações
temporais entre diversidade e supostas variáveis explicativas são as
evidências mais fortes que se pode obter para se que se reconheçam
relações causais. Ainda, estudos temporais de
diversidade são essenciais para prever mudanças futuras na estrutura
das comunidades e no funcionamento dos ecossistemas. 

No entanto, apesar das SADs serem uma ferramenta e um objeto de estudo
consagrado nas pesquisas sobre diversidade \citep{McGill2007}, 
raríssimos estudos avaliaram SADs empíricas
obtidas a partir de técnicas de amostragem que levaram explicitamente
em conta o problema da detecção imperfeita. Assim, conforme o
enunciado do problema geral deste projeto, os padrões empíricos de
diversidade que embasaram as teorias clássicas de ecologia de
comunidades confundem o processo ecológico – o real objeto de
interesse – com o processo observacional. É incerto o
quanto a confusão pode enviesar nossa percepção dos padrões - e dos
processos - que estruturam comunidades biológicas. Assim, esta meta do
projeto tem dois focos: a) investigar como a diversidade de uma
assembleia de borboletas (Nymphalidae, Ithomiinae) muda ao longo do
tempo, testando as previsões de modelos de estruturação de comunidades
que se baseiam em processos distintos (neutro/nicho);  b) comparar os
padrões empíricos de diversidade de borboletas oriundos de dados crus
com os de modelos que consideram detecção imperfeita. 

Há muitas teorias e modelos para explicar os padrões de diversidade
observados nas comunidades biológicas. Não obstante, o problema é que
a maior parte destas teorias e modelos faz uma única previsão: que as
SADs empíricas terão uma forma côncava. Porém, apenas prever uma curva
côncava não pode servir como teste decisivo entre teorias porque todas
as teorias fazem essa previsão. Ter uma teoria
que produz uma SADs côncava realista, mesmo uma que se ajuste muito bem
a um conjunto de dados empíricos, não é suficiente para validar uma
teoria. Além disso, uma dada formulação
matemática de uma SADs (i.e., um padrão) pode ser gerada por diversos
processos distintos, portanto ajuste aos dados per se não pode ser uma
prova definitiva da importância de um processo em particular 
\citep{Pielou1975,McGill2003}. 

Uma abordagem promissora para contornar esta limitação é dada por
teorias que podem simultaneamente produzir SADs côncavas e outros
padrões macroecológicos a partir de um pequeno conjunto de premissas, 
tal como a Teoria Neutra \citep{Caswell1976,Hubbell2001}. 
Os modelos de dinâmica neutra de comunidades geram previsões sobre
curvas espécies-tempo \citep{adler2005}, variações na composição de
espécies, \emph{turnover} temporal de espécies, e modelos para 
SADs \citep{Hubbell2001}. Não obstante, a maioria dos testes
empíricos desses modelos têm se limitado a simplesmente examinar a
consistência da SADs de comunidades locais com o modelo previsto,
que é uma das variações da distribuição multinomial de soma zero (ZSM). Alguns dos
parâmetros da ZSM (tamanho da
metacomunidade, taxa de especiação e taxa de migração) são muito
difíceis ou impossíveis de se medir diretamente
\citep{wootton2005}. Por isso, geralmente são mantidos livres (i.e., estimados a
partir de ajuste aos dados) permitindo que a ZSM assuma uma ampla gama
de formas funcionais e se ajuste bem a diferentes SADs empíricas. 
Assim, há hoje um consenso de que esse é um teste fraco de
neutralidade em comunidades \citep{McGill2003,leigh2007}. 

Além de apenas comparar o ajuste de SADs empíricas com SADs previstas a
partir de parâmetros livres, a maioria dos testes da Teoria Neutra
considerou as SADs (ou outras medidas) num dado momento, sem
levar em conta a dimensão temporal e abrindo mão,
portanto, de todo o poder preditivo que um modelo de dinâmica pode oferecer \citep{leigh2007}. 

Avaliaremos previsões simultâneas da Teoria Neutra
- SADs, \emph{turnover} na composição de espécies e curvas espécies-tempo - em
diferentes resoluções temporais, comparando-as com padrões empíricos
de assembleias de borboletas nectarívoras de subosque obtidos através
de técnicas que consideram ou não detecção imperfeita. Além disso,
também testaremos as previsões obtidas de modelos baseados na Teoria
Neutra, mas que não assumem equivalência entre indivíduos,
representando, portanto, previsões deduzidas de processos baseados em
nicho. O fato de testar simultaneamente diferentes previsões temporais
dos modelos - e não apenas a forma estática da SADs prevista - torna
nosso teste da Teoria Neutra mais rigoroso. 

\subsubsection*{Procedimentos}
\label{sec:procedimentos}

O sistema de estudo deste componente são assembleias de borboletas
nectarívoras de subosque (Nymphalidae, Ithomiinae) distribuídas em
manchas florestais na Cidade Universitária Armando
Sales Oliveira (CUASO), em São Paulo. São populações permanentes de
pelo menos 18 espécies que ocorrem em fragmentos florestais urbanos no campus
da USP e entorno. A facilidade de acesso possibilita censos repetidos e de longo termo. Essas
borboletas também são fáceis de capturar e marcar, e têm mais de uma
geração por ano. A dinâmica anual destas assembleias varia com a
sazonalidade na umidade. Nas estações secas e quentes (outono e
primavera), as espécies se agrupam em bolsões, que são agregados de
indivíduos em locais úmidos próximos ao chão da mata, 
os quais podem ter centenas ou até milhares de indivíduos
\citep{canals2003}. 
No verão, com o calor e aumento da umidade, as
borboletas se dispersam para se acasalar e ovopositar (Gustavo
Accacio, comunicação pessoal). A assembleia das borboletas Ithomiinae
forma dois anéis miméticos Müllerianos, cujas espécies sofreriam
pressão seletiva a favor da manutenção de suas similaridades. Assim,
essa assembleia deve se aproximar da premissa 
de equivalência entre as espécies da teoria neutra. 

Os padrões empíricos (SADs, \emph{turnover} e curvas espécies-tempo) e
parâmetros demográficos de interesse (recrutamento, sobrevivência e
migração) serão estimados através de técnicas de
marcação-recaptura. As borboletas serão capturadas com rede
entomológica e identificadas com marcas individuais feitas com caneta permanente na face ventral de uma
das asas. Será utilizado o delineamento
robusto de Pollock (multi-estado), o qual combina modelos de
populações fechadas e abertas e considera detecção imperfeita 
\citep{pollock2002,williams2002}. O censo terá duração de pelo menos
três anos.
Seguindo recomendações de
\citet{williams2002}, modelos de populações fechadas serão usados
para estimar abundâncias; modelos de populações abertas serão usados
para estimar sobrevivência; recrutamento será estimado com o uso
conjunto de modelos abertos e fechados; e migração será estimada como
a probabilidade de um indivíduo passar de um estado (bolsão em uma
dada mancha florestal) para outro. Ainda, serão comparados modelos que
assumem variação nas taxas de migração, recrutamento e sobrevivência
ao longo das diferentes fases do ciclo de formação e dissipação dos
bolsões – conforme hipótese levantada por um estudo anterior no mesmo
sistema - e modelos que assumem que essas taxas são
constantes ao longo do tempo. Os parâmetros populacionais acima
descritos, bem como a probabilidade de captura, serão calculados com o
programa MARK \citep{white2001}.

As estimativas dos parâmetros demográficos serão então usadas em
simulações computacionais de processos estocásticos de nascimento
morte e dispersão, com e sem diferenciação de nichos. As rotinas para
estas simulações estão em desenvolvimento pelo meu laboratório
(\url{http://pilaboratory.github.io/TWoLife/}), e também são um produto
deste projeto. As simulações computacionais fornecerão valores
previstos por modelos de nicho e neutras de abundâncias, taxas de
\emph{turnover} e curvas espécies-tempo (curvas do coletor) e outros padrões
das comunidades. Cada previsão gerada nas simulações - com sua
respectiva medida de incerteza - será comparada com os dados empíricos
a fim de se avaliar qual dos dois tipos de modelo (neutro ou nicho) descreve
melhor aos dados \citep{hartig2011}. 

\subsection{Aves: complexidade de habitats e diversidade}
\label{sec:compl-de-habit}

Em contraposição às teorias neutras, a teoria clássica de ecologia de
comunidades considera a partilha de nichos fundamental para a
diversidade de espécies em comunidades naturais 
\citep{diamond1978,tilman1982,chase2003}. 
Essa visão clássica é
sintetizada no modelo bem conhecido de diversidade de
\citet{MacArthur1972}, em que a diversidade de espécies $D$ é uma função da diversidade de
recursos disponíveis $R$, amplitude de nicho média $W$, número de
potenciais competidores $C$ e sobreposição do nicho média entre as
espécies $O$: 

\begin{displaymath}
  D \ = \  \frac{1+C}{O} \ \frac{R}{W}
\end{displaymath}

Sob a premissa de que as espécies competem por recursos
limitantes, esse modelo prevê que a diversidade de espécies na
comunidade seria maior quanto maior a diversidade de recursos, menor a
amplitude média de nicho e maior a sobreposição média de nichos dessas
espécies.  

Há muitas evidências de que a diversidade de espécies é
fortemente influenciada pela heterogeneidade ambiental 
\citep{cody1985,rosenzweig1995,hurlbert2003}. 
A explicação proposta é
que quanto mais heterogêneo o ambiente, maior a quantidade potencial
de nichos a serem ocupados pelas espécies 
\citep{hurlbert2004,diaz2006}. A diversidade estrutural da vegetação, é uma das medidas mais
utilizadas para se quantificar a heterogeneidade espacial no ambiente
\citep{tews2004}. Quanto mais complexa é a estrutura da vegetação,
maior a disponibilidade e diversidade de recursos, o que permite a
segregação das espécies espacialmente e diminui a competição entre
elas \citep{macarthur1958}.

O objetivo desse componente da pesquisa é avaliar, por meio do teste
das previsões derivadas do modelo de MacArthur, a influência da
diversidade de recursos, do uso e da partilha destes recursos pelas
espécies sobre a diversidade de espécies na comunidade. A variação da
complexidade da vegetação por meio de um gradiente ambiental de
heterogeneidade pode ajudar a entender qual o seu papel na
determinação da diversidade de espécies. O Cerrado é um bom modelo
para este estudo por incluir em um mesmo regime climático 
fitofisionomias vegetais que variam
marcadamente quanto à estrutura, de campos, savanas a
florestas.  

As aves são um dos grupos biológicos mais utilizados no teste e
proposição de teorias ecológicas. A maioria das espécies 
são relativamente fáceis de se observar e identificar em
campo, proporcionando ótimo conhecimento
acerca da biologia e ecologia das espécies. As comunidades de aves do
Cerrado são bem conhecidas, mas sua análise com
métodos que considerem a probabilidade de detecção das espécies não
foi explorada até o momento. Acreditamos tratar-se de um refinamento
importante e bastante promissor para ser incorporado em estudos de
diversidade \citep{gu2004}. Um de nossos pressupostos, que
testaremos no componente de síntese deste projeto, é que teremos estimativas
de ocupação de estratos e de fisionomias e de diversidade mais
confiáveis quando consideramos a probabilidade de detecção das
espécies.

\subsubsection*{Procedimentos}
\label{sec:aves_proc}

Realizaremos os estudo no Parque Nacional Grande Sertão Veredas,
situado no noroeste de Minas Gerais, município de Chapada
Gaúcha. É um local propício para o estudo da relação entre diversidade
de espécies e diferentes tipos de vegetação, por ser um dos maiores e
mais preservados remanescentes do bioma Cerrado e por abrigar uma
grande diversidade de fitofisionomias vegetais, todas 
em grande extensão dentro dos limites do parque
(Plano de manejo PARNA Grande Sertão Veredas, \url{http://www.icmbio.gov.br}).
 
Serão amostrados 24 transectos com distância mínima de 500m entre eles,
distribuídos igual e sistematicamente entre três fitofisionomias de
cerrado (campo limpo ou campo cerrado, cerrado \emph{sensu stricto}) e
cerradão, de forma
a representar um gradiente de aumento de complexidade da vegetação. A
estrutura da vegetação ao longo dos trasectos será mensurada a partir
do método de \citet{wiens1981}. A detecção visual dos animais
será necessária para que se possa realizar a estimativa da altura que
o indivíduo foi avistado, a qual será usada para estimar o uso dos estratos de
vegetação pelas espécies e a amplitude e sobreposição de nicho das
espécies. A estimativa da altura será feita visualmente com a ajuda de
um telêmetro e de um clinômetro, que possibilitam a medição da
distância de um ponto a partir da emissão de um laser e calcula o
ângulo entre duas distâncias, respectivamente. As amostragens serão
realizadas entre 06:30 e 10:30 da manhã e entre 14:00 e 18:00 da
tarde, que são os períodos de maior atividade das aves, mas poderão
ser ajustados devido à variação da luminosidade ao longo do ano. Todos
os transectos serão amostrados ao menos oito dias em duas campanhas de
campo. Análises parciais dos dados avaliarão indicarão se ajustes ao
delineamento robusto são necessárias. 

As probabilidades de ocupação e de detecção serão estimadas com os
modelos de ocupação com detecção imperfeita para comunidades \citep{mackenzie2003},
em que as probabilidades são funções de covariáveis de interesse ou
que se pretenda descontar. Os modelos serão comparados pelo
procedimento de seleção de modelos pelo critério de Akaike
\citep{Burnham2002}. 

Um primeiro conjunto de modelos será usado para comparar as riquezas
de espécies entre fisionomias. Para isso, além de outras covariáveis
de detecção e ocorrência
que se façam necessárias (\emph{e.g.} dia da coleta, condições de
tempo, horário), confrontaremos modelos com e sem covariáveis de
complexidade da vegetação.

Um segundo conjunto de modelos será usado para testar diretamente
mudanças de amplitude e sobreposição de uso da vegetação entre
fisionomias. Para isso, os modelos anteriores serão adaptados para
incluir o estrato da vegetação em que cada indvíduo foi registrado.
Após a seleção do modelo mais plausível para representar os dados, as
probabilidades de ocupação das espécies corrigidas pela sua
probabilidade de detecção serão usadas para o cálculo da amplitude e
sobreposição de uso dos estratos usando a matriz de probabilidade de
ocupação. Nessa matriz, as espécies são representadas nas linhas, os
habitats nas colunas e as probabilidades de ocorrência das espécies
por habitats são os elementos da matriz. Com essa matriz, a amplitude
média e sobreposição média de nichos da assembleia de especies é
calculada por diferentes partições aditivas da informação total de
Shannon da matriz \citep{Pielou1972}. Finalmente, para o teste 
das previsões do modelo de diversidade de
MacArthur, usaremos modelos lineares generalizados em que a variável
dependente é a riqueza de espécies corrigida pela detectabilidade, e
as variáveis preditoras serão a diversidade de estratos da vegetação,
e os inversos das amplitudes e a sobreposições médias de nichos em
cada ponto, conforme a expressão de \citet{MacArthur1972}.  Estes modelos
serão comparados com concorrentes sem o efeito de cada uma e de
todas as variáveis preditoras, por meio da seleção de modelos com
critério de informação de Akaike. 

\subsection{Baleias: dinâmica populacional e flutuações climáticas}
\label{sec:dinam-popul-de}

O crescimento populacional humano está afetando a dinâmica de
populações naturais pela perda de habitats, exploração direta ou pelas
mudanças climáticas. Uma das chaves para entender o problema são
modelos dinâmicos que acoplem, de maneira mecanística, a demografia
das espécies às mudanças causadas pelo homem. A construção de modelos
quantitativos de dinâmica de populações é um componente central da
ecologia e da biologia da conservação \citep{sutherland2008}. 
Esses modelos são essenciais para compreender e intervir nos
principais problemas ambientais, como as invasões biológicas
\citep{gurevitch2011} e a extinção de espécies \citep{morris2002}. 
Assim, investigar processos subjacentes e reguladores do
crescimento populacional tem sido um dos focos da ecologia
populacional \citep{rockwood2006}. Entre tais processos
estão fatores que afetam a sobrevivência e fecundidade, como a
predação e disponibilidade de presas. Dessa forma, o crescimento
populacional é resultado de vários fatores que podem operar mais ou
menos independentemente, e responder em diferentes escalas 
de tempo \citep{croxall1992}.

A baleia-jubarte (\emph{Megaptera novaeangliae}) é uma espécie
migratória anual, que alimenta-se nas regiões polares mas reproduz-se
em regiões tropicais e subtropicais. Foi quase extinta com a caça comercial
que aconteceu na primeira metade do século passado no Hemisfério Sul
\citep{findlay2001}, quando restavam menos de 5\% de sua população
original \citep{clapham1999}. Atualmente a população de
baleias-jubarte que reproduz na costa brasileira, e se alimenta ao
redor da Antártica, está crescendo \citep{ward2006}. Trata-se de um
crescimento exponencial ou próximo dele, que é raramente observada na
natureza e tende a durar pouco tempo, até opere algum fator
limitante \citep{rockwood2006}. Estudar o crescimento populacional da
baleia-jubarte nesse momento é uma rara oportunidade para estimar o
potencial biológico de crescimento de uma espécie de grande porte e
que esteve no limiar de extinção.

Todos os anos, durante o inverno austral, o continente antártico
aumenta drasticamente de tamanho, quando um ``mar de gelo'', ou
banquisa, se forma ao seu redor. A banquisa tem
a espessura de cerca de um metro e tem um papel importante na cadeia
trófica antártica, afetando o krill (\emph{Euphasia superba}), uma espécie de
crustáceo que é chave no ecossistema polar por ser extremamente abundante
e consumida por várias espécies \citep{everton2000}. O krill é um
consumidor primário e generalista, responsável por boa parte da
transferência de energia entre os produtores primários e os níveis
mais altos da cadeia trófica antártica. Em fases críticas de seu
desenvolvimento, o krill se alimenta de
micro-organismos que vivem na superfície inferior do gelo. 
O derretimento do mar de gelo na primavera propicia uma maior
produtividade fitoplanctônica, que resulta em maior disponibilidade de
alimento para o krill \citep{nicol2006}.

A densidade de krill no Oceano Atlântico Sul Ocidental é
correlacionada com a extensão e duração do mar de gelo no inverno
anterior \citep{loeb1997,atkinson2004}. De uma maneira
geral, a extensão do mar de gelo da Antártica já reduziu nas últimas
décadas \citep{trivelpiece2011}, uma tendência que deve manter-se 
\citep{ipcc2007}. Por afetar diretamente a densidade de krill, as
projeções são de profundos efeitos sobre a cadeia trófica Antártica
\citep{nicol2008}. 
Assim, é urgente o entendimento dos fatores ambientais que governam a
dinâmica de populações nesses sistemas e sua relação com o clima, 
seja de maneira direta ou indireta
\citep{boggs2012}. 
As grandes baleias têm respostas indiretas às mudanças
climáticas. Menores taxas reprodutivas foram sugeridas para baleias do
gênero \emph{Eubalaena} após períodos quentes, e consequente
menor disponibilidade de presas em suas áreas de alimentação em ambos
os hemisférios Norte e Sul \citep{greene2004,leaper2006}. 
As populações de pinguins no mar de Scotia e Península
Antártica, que não foram caçadas e, portanto, representam melhor
mudanças que acontecem no ecossistema Antártico, diminuiram nas
últimas décadas \citep{trivelpiece2011}. Se menos gelo
significa menos krill, e a extensão do mar de gelo diminuiu nas
últimas décadas e espera-se que continue diminuindo com as mudanças
climáticas, podemos prever um efeito \emph{bottom-up}, com consequências
negativas para os predadores de krill, como a baleia-jubarte e outros
organismos.

Deste modo, nossa hipótese deste estudo de caso é que fatores
climáticos ligados à extensão máxima do mar de gelo no inverno afetem
indiretamente as taxas vitais da baleia-jubarte, possivelmente com
lapsos de tempo de um ou mais anos. Partimos do princípio de que
sinais da variabilidade climática subdecadal produzem respostas de
curto a médio prazo nas populações de baleias-jubarte que reproduzem
em águas brasileiras. Entre as respostas esperadas das populações de
baleia-jubarte frente a uma menor extensão de gelo estariam a
diminuição da taxa de sobrevivência de adultos e/ou filhotes, e um
aumento do intervalo entre partos das fêmeas.

\subsubsection*{Procedimentos}
\label{sec:procedimentos-1}

Serão usados dois bancos de dados de longo prazo coletados na área de
reprodução da baleia-jubarte na costa brasileira: (1) cruzeiros de
pesquisa na região de Abrolhos entre 1989 e 2011, com esforço de mais de
30.000 milhas náuticas e mais de 4.000 baleias identificadas através
de marcas naturais até 2010 \citep{wedekin2010}; e (2) sobrevoos
nas costas dos Estados do Espírito Santo e Bahia entre 2001 e 2011
 de aproximadamente 18.000 milhas náuticas percorridas e 1.891 grupos de
baleia-jubarte observados \citep{andriolo2010}. A matriz de
históricos individuais das baleias-jubarte observadas em mais de 20
anos de pesquisa no Banco dos Abrolhos, e outras informações
associadas (sexo, tamanho, presença de filhote) será o principal banco
de dados utilizado na presente proposta. Estes bancos de dados já
estão disponíveis para este projeto e foram coletados ao longo dos
últimos 20 anos com a participação de Leonardo Wedekin, que é
pesquisador pos-doc em meu  laboratório.

Para obtenção das informações climáticas serão usados os dados
coletados através de sensoriamento remoto disponíveis em diferentes
instituições de pesquisa no mundo. Dentre as informações que podemos
obter estão séries históricas sobre: (1) as temperaturas superficiais
do mar e do ar; (2) índice de oscilação climática (SOI - \emph{Southern
Oscillation Index}), que mede a força de eventos de El Niño; (3)
extensão e duração do mar de gelo a diferentes concentrações
(densidade do gelo); dentre outras. Estas variáveis serão incorporadas
aos modelos de dinâmica populacional como preditoras das variações nas
taxas vitais.

As taxas vitais da baleia-jubarte (e.g., sobrevivência, intervalo
reprodutivo, taxa de crescimento da população) serão modeladas em
função das variáveis climáticas, usando modelos probabilísticos
como o modelo de Cormack-Jolly-Seber, usado para estimar a
sobrevivência, ou o modelo de Pradel, usado para estimar a taxa de
crescimento da população \citep{lebreton1992}. Além de modelar os
processos de detecção e do processo ecológico de interesse
separadamente, covariáveis de cada campanha serão incluídas nos
modelos de maneira a corrigir potenciais distorções amostrais (e.g.,
maior ou menor esforço de campo, condições climáticas) ou diferenças
individuais (e.g., machos e fêmeas, diferentes comportamentos de
movimentação ou fidelidade ao sítio) que afetam a detectabilidade ou a
sobrevivência. Os modelos de marcação-recaptura também podem expressar
as taxas vitais (e.g., intervalo reprodutivo das fêmeas, recrutamento,
sobrevivência dos adultos) como funções lineares de variáveis
climáticas (e.g., índices de oscilação climática, extensão do mar de
gelo e temperatura média para as regiões onde as baleias-jubarte que
reproduzem no Brasil se alimentam). Com essa abordagem serão testados
modelos com diferentes variáveis climáticas e diferentes lapsos de
tempo, pois se espera que algumas respostas demográficas tenham
atrasos de um ou mais anos \citep[e.g.][]{leaper2006}. Diferentes
modelos incluindo covariáveis serão construídos usando o programa
livre MARK \citep{White1999}. Alguns modelos mais complexos e
suas variações deverão ser construídos em ambientes de programação
como o \R \  \citep{R2012}.

O primeiro e importante passo para prever as consequências das
mudanças climáticas sobre uma espécie é estimar as taxas vitais que
regulam o crescimento populacional e identificar as conexões destas
taxas vitais com variações ambientais \citep{regehr2010}. 
Estimativas confiáveis de parâmetros demográficos como a
sobrevivência, recrutamento ou tamanho da população são necessárias
para projeções populacionais \citep{white2002}. Parâmetros
demográficos estimados por modelos estatísticos serão incluídos em
simulações que levam em consideração o acaso (estocasticidade
demográfica e ambiental) para projetar a trajetória da população sobre
diferentes condições (cenários) de mudança climática. 
% O primeiro passo
% é escolher algum modelo que descreva o parâmetro demográfico que se
% deseja projetar. O segundo passo é repetir várias vezes as projeções
% por simulações computacionais considerando um determinado horizonte de
% tempo e um estado inicial observado (como o tamanho da população atual
% e sua distribuição por sexo e idade). 
As trajetórias hipotéticas das populações
devem permitir a variação dos seus parâmetros de acordo com alguma
distribuição definida a priori, incorporando as variações temporais e
demográficas nas trajetórias simuladas (chamado de ruído branco). O
último passo é achar um valor central que sumarie estas trajetórias
estocásticas simuladas, como a mediana, e seus intervalos de
confiança.

\subsection{Síntese: sensibilidade à detecção imperfeita}

Para atingir o obejetivo deste componente de síntes, os testes das hipóteses de cada estudo de
caso serão repetidos com estimativas de abundância, ocorrência e taxas
vitais calculadas sem levar em consideração a detecção
imperfeita. Essas ``estimativas ingênuas'' \citep[\emph{naive
  estimates}][]{williams2002} 
podem ser obtidas usando taxas brutas equivalentes, ou
fixando os parâmetros de detectabilidade como igual a $1$, ou seja, com
a detecção perfeita dos indivíduos ou espécies. Este componente gerará
um artigo metodológico, que avaliará em que extensão o uso de métodos
com detecção imperfeita afetam as conclusões qualitativas e
quantitativas dos três estudos de caso, buscando generalizar
considerações gerais das vantagens e desvantagens desses métodos em
ecologia. 

\section{Disseminação e avaliação}
\label{sec:diss-e-aval}

O principal meio de divulgação dos resultados deste projeto serão
artigos  em periódicos indexados. Cada estudo de caso
deve gerar pelo menos dois artigos em periódicos Qualis A da área de
biodiversidade. Uma publicação com as comparações entre estimativas
convencionais e estimativas considerando a detecção imperfeita também
será gerada, para o periódico \emph{Methods in Ecology and Evolution}
(Qualis A1).

Outro produto deste projeto é a disciplina ``Métodos estatísticos em
ecologia populacional'' que será ministrada anualmente para a
Pós-graduação em Ecologia da USP. (\url{http://ecologia.ib.usp.br/bie5703}),
com participação de toda a equipe deste projeto. O projeto também
resultará em uma biblioteca de simulação de dinâmicas populacionais
estocásticas em C$^{++}$ com interface para o ambiente R
(\url{http://pilaboratory.github.io/TWoLife/}).

\section{Cronograma de Execução}

  \begin{tabular}{|p{6cm}|c c c c c c|}
  \hline
  \multicolumn{1}{|c|}{ETAPA}&\multicolumn{6}{c|}{SEMESTRE}\\
  &1&2&3&4&5&6 \\
   \hline
   Simpósios com toda a equipe &X &X &X &X &X &X  \\
   Disciplina de pós-graduação &X & &X & &X &  \\
   Coleta de dados borboletas &X &X &X &X &X &  \\
   Coleta de dados de aves  &X & &X & &X &  \\
   Biblioteca de simulação em C$^{++}$ &X &X &X &X &X &  \\
   Análise de dados  & & &X &X &X &X  \\
   Síntese das análises  & & & & &X &X  \\
   Redação de artigos & & & & X&X &X  \\
   \hline
 \end{tabular}


\bibliographystyle{ecol_let}
\bibliography{/home/paulo/work/resources/bib/geral}
\end{document}
